%!TEX root = ts.tex

\rSec0[async]{Asynchronous model}


\indexlibrary{\idxhdr{experimental/executor}}%
\rSec1[async.synop]{Header \tcode{<experimental/executor>} synopsis}

\indexlibrary{\idxcode{uses_executor_v}}%
\indexlibrary{\idxcode{associated_allocator_t}}%
\indexlibrary{\idxcode{associated_executor_t}}%
\begin{codeblock}
namespace std {
namespace experimental {
namespace net {
inline namespace v1 {

  template<class CompletionToken, class Signature>
    class async_result;

  template<class CompletionToken, class Signature>
    struct async_completion;

  template<class T, class ProtoAllocator = allocator<void>>
    struct associated_allocator;

  template<class T, class ProtoAllocator = allocator<void>>
    using associated_allocator_t = typename associated_allocator<T, ProtoAllocator>::type;

  // \ref{async.assoc.alloc.get}, get_associated_allocator:

  template<class T>
    associated_allocator_t<T> get_associated_allocator(const T& t) noexcept;
  template<class T, class ProtoAllocator>
    associated_allocator_t<T, ProtoAllocator>
      get_associated_allocator(const T& t, const ProtoAllocator& a) noexcept;

  enum class fork_event {
    prepare,
    parent,
    child
  };

  class execution_context;

  class service_already_exists;

  template<class Service> Service& use_service(execution_context& ctx);
  template<class Service, class... Args> Service&
    make_service(execution_context& ctx, Args&&... args);
  template<class Service> bool has_service(execution_context& ctx) noexcept;

  struct executor_arg_t { };
  constexpr executor_arg_t executor_arg = executor_arg_t();

  template<class T, class Executor> struct uses_executor;

  template<class T, class Executor>
    constexpr bool uses_executor_v = uses_executor<T, Executor>::value;

  template<class T, class Executor = system_executor>
    struct associated_executor;

  template<class T, class Executor = system_executor>
    using associated_executor_t = typename associated_executor<T, Executor>::type;

  // \ref{async.assoc.exec.get}, get_associated_executor:

  template<class T>
    associated_executor_t<T> get_associated_executor(const T& t) noexcept;
  template<class T, class Executor>
    associated_executor_t<T, Executor>
      get_associated_executor(const T& t, const Executor& ex) noexcept;
  template<class T, class ExecutionContext>
    associated_executor_t<T, typename ExecutionContext::executor_type>
      get_associated_executor(const T& t, ExecutionContext& ctx) noexcept;

  template<class T, class Executor>
    class executor_binder;

  template<class T, class Executor, class Signature>
    class async_result<executor_binder<T, Executor>, Signature>;

  template<class T, class Executor, class ProtoAllocator>
    struct associated_allocator<executor_binder<T, Executor>, ProtoAllocator>;

  template<class T, class Executor, class Executor1>
    struct associated_executor<executor_binder<T, Executor>, Executor1>;

  // \ref{async.bind.executor}, bind_executor:

  template<class Executor, class T>
    executor_binder<decay_t<T>, Executor>
      bind_executor(const Executor& ex, T&& t);
  template<class ExecutionContext, class T>
    executor_binder<decay_t<T>, typename ExecutionContext::executor_type>
      bind_executor(ExecutionContext& ctx, T&& t);

  class system_context;
  using system_executor = system_context::executor_type;

  class bad_executor;

  class executor;

  bool operator==(const executor& a, const executor& b) noexcept;
  bool operator==(const executor& e, nullptr_t) noexcept;
  bool operator==(nullptr_t, const executor& e) noexcept;
  bool operator!=(const executor& a, const executor& b) noexcept;
  bool operator!=(const executor& e, nullptr_t) noexcept;
  bool operator!=(nullptr_t, const executor& e) noexcept;

  // \ref{async.dispatch}, dispatch:

  template<class CompletionToken>
    @\DEDUCED@ dispatch(CompletionToken&& token);
  template<class Executor, class CompletionToken>
    @\DEDUCED@ dispatch(const Executor& ex, CompletionToken&& token);
  template<class ExecutionContext, class CompletionToken>
    @\DEDUCED@ dispatch(ExecutionContext& ctx, CompletionToken&& token);

  // \ref{async.post}, post:

  template<class CompletionToken>
    @\DEDUCED@ post(CompletionToken&& token);
  template<class Executor, class CompletionToken>
    @\DEDUCED@ post(const Executor& ex, CompletionToken&& token);
  template<class ExecutionContext, class CompletionToken>
    @\DEDUCED@ post(ExecutionContext& ctx, CompletionToken&& token);

  // \ref{async.defer}, defer:

  template<class CompletionToken>
    @\DEDUCED@ defer(CompletionToken&& token);
  template<class Executor, class CompletionToken>
    @\DEDUCED@ defer(const Executor& ex, CompletionToken&& token);
  template<class ExecutionContext, class CompletionToken>
    @\DEDUCED@ defer(ExecutionContext& ctx, CompletionToken&& token);

  template<class Executor>
    class strand;

  template<class Executor>
    bool operator==(const strand<Executor>& a, const strand<Executor>& b);
  template<class Executor>
    bool operator!=(const strand<Executor>& a, const strand<Executor>& b);

  template<class ProtoAllocator = allocator<void>>
    class use_future_t;

  constexpr use_future_t<> use_future = use_future_t<>();

  template<class ProtoAllocator, class Result, class... Args>
    class async_result<use_future_t<ProtoAllocator>, Result(Args...)>;

  template<class Result, class... Args, class Signature>
    class async_result<packaged_task<Result(Args...)>, Signature>;

} // inline namespace v1
} // namespace net
} // namespace experimental

  template<class Allocator>
    struct uses_allocator<experimental::net::v1::executor, Allocator>
      : true_type {};

} // namespace std
\end{codeblock}



\rSec1[async.reqmts]{Requirements}


\rSec2[async.reqmts.proto.allocator]{Proto-allocator requirements}

\pnum
A type \tcode{A} meets the proto-allocator requirements if \tcode{A} is \tcode{CopyConstructible} (\CppXref{copyconstructible}), \tcode{Destruct\-ible} (\CppXref{destructible}), and \tcode{allocator_traits<A>::rebind_alloc<U>} meets the allocator requirements (\CppXref{allocator.requirements}), where \tcode{U} is an object type. \begin{note} For example, \tcode{allocator<void>} meets the proto-allocator requirements but not the allocator requirements. \end{note} No comparison operator, copy operation, move operation, or swap operation on these types shall exit via an exception.



\rSec2[async.reqmts.executioncontext]{Execution context requirements}

\indexlibrary{\idxcode{execution_context}}%
\pnum
A type \tcode{X} meets the \tcode{ExecutionContext} requirements if it is publicly and unambiguously derived from \tcode{execution_context}, and satisfies the additional requirements listed below.

\pnum
In Table~\ref{tab:async.reqmts.executioncontext.requirements}, \tcode{x} denotes a value of type \tcode{X}.

\indextext{requirements!\idxcode{ExecutionContext}}%
\indextext{\idxcode{ExecutionContext}}%
\begin{libreqtab3}
{ExecutionContext requirements}
{tab:async.reqmts.executioncontext.requirements}
\\ \topline
\lhdr{expression}  &
\chdr{return type}  &
\rhdr{assertion/note pre/post-condition} \\ \capsep
\endfirsthead
\continuedcaption\\
\hline
\lhdr{expression}  &
\chdr{return type}  &
\rhdr{assertion/note pre/post-condition} \\ \capsep
\endhead

\tcode{X::executor_type}  &
A type satisfying the general requirements on executors (P0443R5) &
  \\ \rowsep

\tcode{x.\~X()}  &
  &
 Destroys all unexecuted function objects that were submitted via an executor object that is associated with the execution context.  \\ \rowsep

\tcode{x.get_executor()}  &
\tcode{X::executor_type}  &
 Returns an executor object that is associated with the execution context.  \\

\end{libreqtab3}



\rSec2[async.reqmts.service]{Service requirements}

\indextext{service}
\indexlibrary{\idxcode{execution_context::service}}%
\pnum
A class is a service if it is publicly and unambiguously derived from \tcode{execution_context::service}, or if it is publicly and unambiguously derived from another service. For a service \tcode{S}, \tcode{S::key_type} shall be valid and denote a type (\CppXref{temp.deduct}), \tcode{is_base_of_v<typename S::key_type, S>} shall be \tcode{true}, and \tcode{S} shall satisfy the \tcode{Destructible} requirements (\CppXref{destructible}).

\pnum
The first parameter of all service constructors shall be an lvalue reference to \tcode{execution_context}. This parameter denotes the \tcode{execution_context} object that represents a set of services, of which the service object will be a member. \begin{note} These constructors may be called by the \tcode{make_service} function. \end{note}

\pnum
A service shall provide an explicit constructor with a single parameter of lvalue reference to \tcode{execution_context}. \begin{note} This constructor may be called by the \tcode{use_service} function. \end{note}

\pnum
\begin{example}

\begin{codeblock}
class my_service : public execution_context::service
{
public:
  using key_type = my_service;
  explicit my_service(execution_context& ctx);
  my_service(execution_context& ctx, int some_value);
private:
  virtual void shutdown() noexcept override;
  ...
};
\end{codeblock}

\end{example}

\pnum
A service's \tcode{shutdown} member function shall destroy all copies of function objects that are held by the service.



\rSec2[async.reqmts.signature]{Signature requirements}

\indextext{requirements!signature}%
\indextext{signature requirements}%
\pnum
A type satisfies the signature requirements if it is a call signature (\CppXref{func.def}).



\rSec2[async.reqmts.associator]{Associator requirements}

\indextext{requirements!associator}%
\indextext{associator}%
\pnum
An associator defines a relationship between different types and objects where, given:

\begin{itemize}
\item
a source object \tcode{s} of type \tcode{S},

\item
type requirements \tcode{R}, and

\item
a candidate object \tcode{c} of type \tcode{C} meeting the type requirements \tcode{R},
\end{itemize}

an associated type \tcode{A} meeting the type requirements \tcode{R} may be computed, and an associated object \tcode{a} of type \tcode{A} may be obtained.

\pnum
An associator shall be a class template that takes two template type arguments. The first template argument is the source type \tcode{S}. The second template argument is the candidate type \tcode{C}. The second template argument shall be defaulted to some default candidate type \tcode{D} that satisfies the type requirements \tcode{R}.

\pnum
An associator shall additionally satisfy the requirements in Table~\ref{tab:async.reqmts.associator.requirements}.
In this table, \tcode{X} is a class template that meets the associator requirements,
\tcode{S} is the source type,
\tcode{s} is a value of type \tcode{S} or \tcode{const S},
\tcode{C} is the candidate type,
\tcode{c} is a (possibly const) value of type \tcode{C},
\tcode{D} is the default candidate type,
and \tcode{d} is a (possibly const) value of type \tcode{D} that is the default candidate object.

\begin{libreqtab3}
{Associator requirements}
{tab:async.reqmts.associator.requirements}
\\ \topline
\lhdr{expression}  &
\chdr{return type}  &
\rhdr{assertion/note pre/post-conditions} \\ \capsep
\endfirsthead
\continuedcaption\\
\hline
\lhdr{expression}  &
\chdr{return type}  &
\rhdr{assertion/note pre/post-conditions} \\ \capsep
\endhead

\tcode{X<S>::type}  &
\tcode{X<S, D>::type}  &
  \\ \rowsep

\tcode{X<S, C>::type}  &
  &
 The associated type.  \\ \rowsep

\tcode{X<S>::get(s)}  &
\tcode{X<S>::type}  &
Returns \tcode{X<S>::get(S, d)}.  \\ \rowsep

\tcode{X<S, C>::get(s, c)}  &
\tcode{X<S, C>::type}  &
 Returns the associated object.  \\

\end{libreqtab3}

\pnum
The associator's primary template shall be defined. A program may partially specialize the associator class template for some user-defined type \tcode{S}.

\pnum
 Finally, the associator shall provide the following type alias and function template in the enclosing namespace:

\begin{codeblock}
template<class S, class C = D> using @\placeholder{X}@_t = typename @\placeholder{X}@<S, C>::type;

template<class S, class C = D>
typename @\placeholder{X}@<S, C>::type get_@\placeholder{X}@(const S& s, const C& c = d)
{
  return @\placeholder{X}@<S, C>::get(s, c);
}
\end{codeblock}

where \tcode{\placeholder{X}} is replaced with the name of the associator class template.
\begin{note} This function template is provided as a convenience, to automatically deduce the source and candidate types. \end{note}



\rSec2[async.reqmts.async]{Requirements on asynchronous operations}

\pnum
This section uses the names \tcode{Alloc1}, \tcode{Alloc2}, \tcode{alloc1}, \tcode{alloc2}, \tcode{Args}, \tcode{CompletionHandler}, \tcode{completion_handler}, \tcode{Executor1}, \tcode{Executor2}, \tcode{ex1}, \tcode{ex2}, \tcode{f}, \placeholder{i}, \placeholder{N}, \tcode{Signature}, \tcode{token}, \tcode{T}$_i$, \tcode{t}$_i$, \tcode{work1}, and \tcode{work2} as placeholders for specifying the requirements below.


\rSec3[async.reqmts.async.concepts]{General asynchronous operation concepts}

\pnum
An initiating function is a function which may be called to start an asynchronous operation. A completion handler is a function object that will be invoked, at most once, with the result of the asynchronous operation.

\pnum
 The life cycle of an asynchronous operation is comprised of the following events and phases:

\begin{itemize}
\item
Event 1: The asynchronous operation is started by a call to the initiating function.

\item
Phase 1: The asynchronous operation is now outstanding.

\item
Event 2: The externally observable side effects of the asynchronous operation, if any, are fully established. The completion handler is submitted to an executor.

\item
Phase 2: The asynchronous operation is now completed.

\item
Event 3: The completion handler is called with the result of the asynchronous operation.
\end{itemize}

\pnum
In this Technical Specification, all functions with the prefix \tcode{async_} are initiating functions.



\rSec3[async.reqmts.async.token]{Completion tokens and handlers}

\indextext{initiating function}%
\pnum
Initiating functions:

\begin{itemize}
\item
are function templates with template parameter \tcode{CompletionToken};

\indextext{completion token}%
\item
accept, as the final parameter, a completion token object \tcode{token} of type \tcode{CompletionToken};

\indextext{completion signature}%
\item
specify a completion signature, which is a call signature (\CppXref{func.def}) \tcode{Signature} that determines the arguments to the completion handler.
\end{itemize}

\indextext{completion handler}%
\pnum
An initiating function determines the type \tcode{CompletionHandler} of its
completion handler function object by performing
\tcode{typename async_result<decay_t<CompletionToken>, Signature>::completion_handler_type}.
The completion handler object \tcode{completion_handler} is initialized with
\tcode{std::forward<Completion\-Token>(token)}.
\begin{note} No other requirements are placed on the type \tcode{CompletionToken}. \end{note}

\pnum
The type \tcode{CompletionHandler} shall satisfy the requirements of \tcode{Destructible} (\CppXref{destructible}) and \tcode{MoveConstructible} (\CppXref{moveconstructible}), and be callable with the specified call signature.

\pnum
In this Technical Specification, all initiating functions specify a
\completionsig element that defines the call signature \tcode{Signature}.
The \completionsig elements in this Technical Specification have named
parameters, and the results of an asynchronous operation are specified in
terms of these names.



\rSec3[async.reqmts.async.return.type]{Deduction of initiating function return type}

\indextext{initiating function!deduction of return type}%
\pnum
The return type of an initiating function is \tcode{typename async_result<decay_t<CompletionToken>, Sig\-nature>::return_type}.

\pnum
For the sake of exposition, this Technical Specification sometimes annotates functions with a return type \DEDUCED. For every function declaration that returns \DEDUCED, the meaning is equivalent to specifying the return type as \tcode{typename async_result<decay_t<CompletionToken>, Signature>::return_type}.



\rSec3[async.reqmts.async.return.value]{Production of initiating function return value}

\indextext{initiating function!production of return value}%
\pnum
 An initiating function produces its return type as follows:

\begin{itemize}
\item
constructing an object \tcode{result} of type \tcode{async_result<decay_t<CompletionToken>, Signature>}, initialized as \tcode{result(completion_handler)}; and

\item
using \tcode{result.get()} as the operand of the return statement.
\end{itemize}

\pnum
\begin{example} Given an asynchronous operation with Completion signature \tcode{void(R1 r1, R2 r2)}, an initiating function meeting these requirements may be implemented as follows:

\begin{codeblock}
template<class CompletionToken>
auto async_xyz(T1 t1, T2 t2, CompletionToken&& token)
{
  typename async_result<decay_t<CompletionToken>, void(R1, R2)>::completion_handler_type
    completion_handler(forward<CompletionToken>(token));

  async_result<decay_t<CompletionToken>, void(R1, R2)> result(completion_handler);

  // initiate the operation and cause completion_handler to be invoked with
  // the result

  return result.get();
}
\end{codeblock}

\pnum
For convenience, initiating functions may be implemented using the \tcode{async_completion} template:

\begin{codeblock}
template<class CompletionToken>
auto async_xyz(T1 t1, T2 t2, CompletionToken&& token)
{
  async_completion<CompletionToken, void(R1, R2)> init(token);

  // initiate the operation and cause init.completion_handler to be invoked
  // with the result

  return init.result.get();
}
\end{codeblock}

\end{example}



\rSec3[async.reqmts.async.lifetime]{Lifetime of initiating function arguments}

\indextext{initiating function!lifetime of arguments}%
\pnum
 Unless otherwise specified, the lifetime of arguments to initiating functions shall be treated as follows:

\begin{itemize}
\item
If the parameter has a pointer type or has a type of lvalue reference to non-const, the implementation may assume the validity of the pointee or referent, respectively, until the completion handler is invoked. \begin{note} In other words, the program must guarantee the validity of the argument until the completion handler is invoked. \end{note}

\item
Otherwise, the implementation does not assume the validity of the argument after the initiating function completes. \begin{note} In other words, the program is not required to guarantee the validity of the argument after the initiating function completes. \end{note} The implementation may make copies of the argument, and all copies shall be destroyed no later than immediately after invocation of the completion handler.
\end{itemize}



\rSec3[async.reqmts.async.non.blocking]{Non-blocking requirements on initiating functions}

\indextext{initiating function!non-blocking requirements}%
\pnum
 An initiating function shall not block (\CppXref{defns.block}) the calling thread pending completion of the outstanding operation.

\pnum
 \begin{note} Initiating functions may still block the calling thread for other reasons. For example, an initiating function may lock a mutex in order to synchronize access to shared data. \end{note}



\rSec3[async.reqmts.async.assoc.exec]{Associated executor}

\indextext{associated executor}%
\pnum
Certain objects that participate in asynchronous operations have an associated executor. These are obtained as specified below.



\rSec3[async.reqmts.async.io.exec]{I/O executor}

\indexlibrary{\idxcode{system_executor}}%
\pnum
An asynchronous operation has an associated executor satisfying the general requirements on executors (P0443R5). If not otherwise specified by the asynchronous operation, this associated executor is an object of type \tcode{system_executor}.

\pnum
 All asynchronous operations in this Technical Specification have an associated executor object that is determined as follows:

\begin{itemize}
\item
If the initiating function is a member function, the associated executor is that returned by the \tcode{get_executor} member function on the same object.

\item
If the initiating function is not a member function, the associated executor is that returned by the \tcode{get_executor} member function of the first argument to the initiating function.
\end{itemize}

\pnum
Let \tcode{Executor1} be the type of the associated executor. Let \tcode{ex1} be a value of type \tcode{Executor1}, representing the associated executor object obtained as described above. \tcode{execution::can_query_v<Executor1, execution::context_t>} shall be \tcode{true}, and \tcode{execution::query(ex1, execution::context_t)} shall yield a value of type \tcode{execution_context&} or of type \tcode{E&}, where \tcode{E} satisifies the \tcode{ExecutionContext}~(\ref{async.reqmts.executioncontext}) requirements.


\rSec3[async.reqmts.async.handler.exec]{Completion handler executor}

\pnum
A completion handler object of type \tcode{CompletionHandler}
has an associated executor
satisfying the general requirements on executors (P0443R5).
The type of this associated executor is
\tcode{associated_executor_t<CompletionHandler, Executor1>}.
Let \tcode{Executor2} be the type
\tcode{associated_executor_t<CompletionHandler, Executor1>}.
Let \tcode{ex2} be a value of type \tcode{Executor2}
obtained by performing \tcode{get_associated_executor(completion_handler, ex1)}.
\tcode{execution::can_query_v<Executor2, execution::context_t>} shall be \tcode{true}, and \tcode{execution::query(ex2, execution::context_t)} shall yield a value of type \tcode{execution_context&} or of type \tcode{E&}, where \tcode{E} satisifies the \tcode{ExecutionContext}~(\ref{async.reqmts.executioncontext}) requirements.



\rSec3[async.reqmts.async.work]{Outstanding work}

\indextext{outstanding work}%
\pnum
Until the asynchronous operation has completed, the asynchronous operation shall maintain:

\begin{itemize}
\item
an executor object \tcode{work1}, initialized as \tcode{execution::prefer(ex1, execution::outstanding_work)}; and

\item
an executor object \tcode{work2}, initialized as \tcode{execution::prefer(ex2, execution::outstanding_work)}.
\end{itemize}



\rSec3[async.reqmts.async.alloc]{Allocation of intermediate storage}

\pnum
Asynchronous operations may allocate memory. \begin{note} Such as a data structure to store copies of the \tcode{completion_handler} object and the initiating function's arguments. \end{note}

\pnum
Let \tcode{Alloc1} be a type, satisfying the \tcode{ProtoAllocator}~(\ref{async.reqmts.proto.allocator}) requirements, that represents the asynchronous operation's default allocation strategy. \begin{note} Typically \tcode{allocator<void>}. \end{note} Let \tcode{alloc1} be a value of type \tcode{Alloc1}.

\pnum
A completion handler object of type \tcode{CompletionHandler} has an associated allocator object \tcode{alloc2} of type \tcode{Alloc2} satisfying the \tcode{ProtoAllocator}~(\ref{async.reqmts.proto.allocator}) requirements. The type \tcode{Alloc2} is \tcode{associated_allocator_t<CompletionHandler, Alloc1>}. Let \tcode{alloc2} be a value of type \tcode{Alloc2} obtained by performing \tcode{get_associated_allocator(completion_handler, alloc1)}.

\pnum
 The asynchronous operations defined in this Technical Specification:

\begin{itemize}
\item
If required, allocate memory using only the completion handler's associated allocator.

\item
Prior to completion handler execution, deallocate any memory allocated using the completion handler's associated allocator.
\end{itemize}

\pnum
 \begin{note} The implementation may perform operating system or underlying API calls that perform memory allocations not using the associated allocator. Invocations of the allocator functions may not introduce data races (See \CppXref{res.on.data.races}). \end{note}



\rSec3[async.reqmts.async.completion]{Execution of completion handler on completion of asynchronous operation}

\pnum
Let \tcode{Args...} be the argument types of the completion signature \tcode{Signature} and let \placeholder{N} be \tcode{sizeof...(Args)}. Let \placeholder{i} be in the range \range{0}{\placeholder{N}}. Let \tcode{T}$_i$ be the $i^\text{th}$ type in \tcode{Args...} and let \tcode{t}$_i$ be the $i^\text{th}$ completion handler argument associated with \tcode{T}$_i$.

\pnum
Let \tcode{f} be a function object, callable as \tcode{f()}, that invokes \tcode{completion_handler} as if by \tcode{completion_handler(forward<T$_0$>(t$_0$), ..., forward<T$_{N-1}$>(t$_{N-1}$))}.

\pnum
If an asynchronous operation completes immediately (that is, within the thread of execution calling the initiating function, and before the initiating function returns), the completion handler shall be submitted for execution as if by performing:
\begin{codeblock}
execution::prefer(
  execution::require(ex2, execution::oneway, execution::single, execution::never_blocking),
    execution::allocator(alloc2)).execute(std::move(f));
\end{codeblock}
Otherwise, the completion handler shall be submitted for execution as if by performing:
\begin{codeblock}
execution::prefer(
  execution::require(work2, execution::oneway, execution::single),
    execution::possibly_blocking, execution::allocator(alloc2)).execute(std::move(f));
\end{codeblock}



\rSec3[async.reqmts.async.exceptions]{Completion handlers and exceptions}

\pnum
 Completion handlers are permitted to throw exceptions. The effect of any exception propagated from the execution of a completion handler is determined by the executor which is executing the completion handler.



\rSec3[async.reqmts.async.composed]{Composed asynchronous operations}

\pnum
In this Technical Specification, a \defn{composed asynchronous operation} is an asynchronous operation that is implemented in terms of zero or more intermediate calls to other asynchronous operations. The intermediate asynchronous operations are performed sequentially. \begin{note} That is, the completion handler of an intermediate operation initiates the next operation in the sequence. \end{note}

An intermediate operation's completion handler shall have an associated executor that is either:

\begin{itemize}
\item
the type \tcode{Executor2} and object \tcode{ex2} obtained from the completion handler type \tcode{CompletionHandler} and object \tcode{completion_handler}; or

\item
an object of an unspecified type satisfying the general requirements on executors (P0443R5), that delegates executor operations to the type \tcode{Executor2} and object \tcode{ex2}.
\end{itemize}

An intermediate operation's completion handler shall have an associated allocator that is either:

\begin{itemize}
\item
the type \tcode{Alloc2} and object \tcode{alloc2} obtained from the completion handler type \tcode{CompletionHandler} and object \tcode{completion_handler}; or

\item
an object of an unspecified type satisfying the ProtoAllocator requirements~(\ref{async.reqmts.proto.allocator}), that delegates allocator operations to the type \tcode{Alloc2} and object \tcode{alloc2}.
\end{itemize}



\rSec1[async.async.result]{Class template \tcode{async_result}}

\indexlibrary{\idxcode{async_result}}%
\pnum
The \tcode{async_result} class template is a customization point for asynchronous operations. Template parameter \tcode{CompletionToken} specifies the model used to obtain the result of the asynchronous operation. Template parameter \tcode{Signature} is the call signature (\CppXref{func.def}) for the completion handler type invoked on completion of the asynchronous operation. The \tcode{async_result} template:

\begin{itemize}
\item
transforms a \tcode{CompletionToken} into a completion handler type that is based on a \tcode{Signature}; and

\item
determines the return type and return value of an asynchronous operation's initiating function.
\end{itemize}

\begin{codeblock}
namespace std {
namespace experimental {
namespace net {
inline namespace v1 {

  template<class CompletionToken, class Signature>
  class async_result
  {
  public:
    using completion_handler_type = CompletionToken;
    using return_type = void;

    explicit async_result(completion_handler_type&) {}
    async_result(const async_result&) = delete;
    async_result& operator=(const async_result&) = delete;

    return_type get() {}
  };

} // inline namespace v1
} // namespace net
} // namespace experimental
} // namespace std
\end{codeblock}

\pnum
The template parameter \tcode{CompletionToken} shall be an object type. The template parameter \tcode{Signature} shall be a call signature (\CppXref{func.def}).

\pnum
Specializations of \tcode{async_result} shall satisfy the \tcode{Destructible} requirements (\CppXref{destructible}) in addition to the requirements in Table~\ref{tab:async.async.result.requirements}. In this table, \tcode{R} is a specialization of \tcode{async_result}; \tcode{r} is a modifiable lvalue of type \tcode{R}; and \tcode{h} is a modifiable lvalue of type \tcode{R::completion_handler_type}.

\indextext{requirements!\idxcode{async_result} specializations}%
\begin{libreqtab3}
{\tcode{async_result} specialization requirements}
{tab:async.async.result.requirements}
\\ \topline
\lhdr{Expression}  &
\chdr{Return type}  &
\rhdr{Requirement} \\ \capsep
\endfirsthead
\continuedcaption\\
\hline
\lhdr{Expression}  &
\chdr{Return type}  &
\rhdr{Requirement} \\ \capsep
\endhead

\tcode{R::completion_handler_type}  &
  &
A type satisfying \tcode{MoveConstructible} requirements (\CppXref{moveconstructible}), An object of type \tcode{completion_handler_type} shall be a function object with call signature \tcode{Signature}, and \tcode{completion_handler_type} shall be constructible with an rvalue of type \tcode{CompletionToken}.  \\ \rowsep

\tcode{R::return_type}  &
  &
\tcode{void}; or a type satisfying \tcode{MoveConstructible} requirements (\CppXref{moveconstructible})  \\ \rowsep

\tcode{R r(h);}  &
  &
  \\ \rowsep

\tcode{r.get()}  &
\tcode{R::return_type}  &
\begin{note} An asynchronous operation's initiating function uses the \tcode{get()} member function as the sole operand of a return statement. \end{note}  \\

\end{libreqtab3}



\rSec1[async.async.completion]{Class template \tcode{async_completion}}

\indexlibrary{\idxcode{async_completion}}%
\pnum
Class template \tcode{async_completion} is provided as a convenience, to simplify the implementation of asynchronous operations that use \tcode{async_result}.

\begin{codeblock}
namespace std {
namespace experimental {
namespace net {
inline namespace v1 {

  template<class CompletionToken, class Signature>
  struct async_completion
  {
    using completion_handler_type = async_result<decay_t<CompletionToken>,
      Signature>::completion_handler_type;

    explicit async_completion(CompletionToken& t);
    async_completion(const async_completion&) = delete;
    async_completion& operator=(const async_completion&) = delete;

    @\seebelow@ completion_handler;
    async_result<decay_t<CompletionToken>, Signature> result;
  };

} // inline namespace v1
} // namespace net
} // namespace experimental
} // namespace std
\end{codeblock}

\pnum
The template parameter \tcode{Signature} shall be a call signature (\CppXref{func.def}).

\begin{itemdecl}
explicit async_completion(CompletionToken& t);
\end{itemdecl}

\begin{itemdescr}
\pnum
\effects If \tcode{CompletionToken} and \tcode{completion_handler_type} are the same type,
binds \tcode{completion_handler} to \tcode{t};
otherwise, initializes \tcode{completion_handler} with \tcode{std::forward<CompletionToken>(t)}.
Initializes \tcode{result} with \tcode{completion_handler}.
\end{itemdescr}

\begin{itemdecl}
@\seebelow@ completion_handler;
\end{itemdecl}

\begin{itemdescr}
\pnum
\ctype \tcode{completion_handler_type\&} if \tcode{CompletionToken} and \tcode{completion_handler_type} are the same type; otherwise, \tcode{completion_handler_type}.
\end{itemdescr}



\rSec1[async.assoc.alloc]{Class template \tcode{associated_allocator}}

\indexlibrary{\idxcode{associated_allocator}}%
\pnum
Class template \tcode{associated_allocator} is an associator~(\ref{async.reqmts.associator}) for the \tcode{ProtoAllocator}~(\ref{async.reqmts.proto.allocator}) type requirements, with default candidate type \tcode{allocator<void>} and default candidate object \tcode{allocator<void>()}.

\begin{codeblock}
namespace std {
namespace experimental {
namespace net {
inline namespace v1 {

  template<class T, class ProtoAllocator = allocator<void>>
  struct associated_allocator
  {
    using type = @\seebelow@;

    static type get(const T& t, const ProtoAllocator& a = ProtoAllocator()) noexcept;
  };

} // inline namespace v1
} // namespace net
} // namespace experimental
} // namespace std
\end{codeblock}

\indextext{requirements!\idxcode{associated_allocator} specializations}%
\pnum
Specializations of \tcode{associated_allocator} shall satisfy the requirements in Table~\ref{tab:async.assoc.alloc.requirements}.
In this table, \tcode{X} is a specialization of \tcode{associated_allocator}
for the template parameters \tcode{T} and \tcode{ProtoAllocator};
\tcode{t} is a (possibly const) value of type \tcode{T};
and \tcode{a} is an object of type \tcode{ProtoAllocator}.

\begin{libreqtab3}
{\tcode{associated_allocator} specialization requirements}
{tab:async.assoc.alloc.requirements}
\\ \topline
\lhdr{Expression}  &
\chdr{Return type}  &
\rhdr{Note}  \\ \capsep
\endfirsthead
\continuedcaption\\
\hline
\lhdr{Expression}  &
\chdr{Return type}  &
\rhdr{Note}  \\ \capsep
\endhead

\tcode{typename X::type}  &
A type meeting the proto-allocator~(\ref{async.reqmts.proto.allocator}) requirements.  &
  \\ \rowsep

\tcode{X::get(t)}  &
\tcode{X::type}  &
Shall not exit via an exception. Equivalent to \tcode{X::get(t, ProtoAllocator())}.  \\ \rowsep

\tcode{X::get(t, a)}  &
\tcode{X::type}  &
 Shall not exit via an exception.  \\

\end{libreqtab3}


\rSec2[async.assoc.alloc.members]{\tcode{associated_allocator} members}

\begin{itemdecl}
using type = @\seebelow@;
\end{itemdecl}

\begin{itemdescr}
\pnum
\ctype \tcode{T::allocator_type} if the \grammarterm{qualified-id} \tcode{T::allocator_type}
is valid and denotes a type~(\CppXref{temp.deduct}).
Otherwise \tcode{ProtoAllocator}.
\end{itemdescr}

\begin{itemdecl}
type get(const T& t, const ProtoAllocator& a = ProtoAllocator()) noexcept;
\end{itemdecl}

\begin{itemdescr}
\pnum
\returns \tcode{t.get_allocator()} if the \grammarterm{qualified-id} \tcode{T::allocator_type}
is valid and denotes a type~(\CppXref{temp.deduct}).
Otherwise \tcode{a}.
\end{itemdescr}



\rSec1[async.assoc.alloc.get]{Function \tcode{get_associated_allocator}}

\indexlibrary{\idxcode{get_associated_allocator}}%
\begin{itemdecl}
template<class T>
  associated_allocator_t<T> get_associated_allocator(const T& t) noexcept;
\end{itemdecl}

\begin{itemdescr}
\pnum
\returns \tcode{associated_allocator<T>::get(t)}.
\end{itemdescr}

\begin{itemdecl}
template<class T, class ProtoAllocator>
  associated_allocator_t<T, ProtoAllocator>
    get_associated_allocator(const T& t, const ProtoAllocator& a) noexcept;
\end{itemdecl}

\begin{itemdescr}
\pnum
\returns \tcode{associated_allocator<T, ProtoAllocator>::get(t, a)}.
\end{itemdescr}



\rSec1[async.exec.ctx]{Class \tcode{execution_context}}

\indexlibrary{\idxcode{execution_context}}%
\pnum
Class \tcode{execution_context} implements an extensible, type-safe, polymorphic set of services, indexed by service type.

\begin{codeblock}
namespace std {
namespace experimental {
namespace net {
inline namespace v1 {

  class execution_context
  {
  public:
    class service;

    // \ref{async.exec.ctx.cons}, construct / copy / destroy:

    execution_context();
    execution_context(const execution_context&) = delete;
    execution_context& operator=(const execution_context&) = delete;
    virtual ~execution_context();

    // \ref{async.exec.ctx.ops}, execution context operations:

    void notify_fork(fork_event e);

  protected:

    // \ref{async.exec.ctx.protected}, execution context protected operations:

    void shutdown() noexcept;
    void destroy() noexcept;
  };

  // service access:
  template<class Service> typename Service::key_type&
    use_service(execution_context& ctx);
  template<class Service, class... Args> Service&
    make_service(execution_context& ctx, Args&&... args);
  template<class Service> bool has_service(const execution_context& ctx) noexcept;
  class service_already_exists : public logic_error { };

} // inline namespace v1
} // namespace net
} // namespace experimental
} // namespace std
\end{codeblock}

\pnum
Access to the services of an \tcode{execution_context} is via three function templates, \tcode{use_service}, \tcode{make_service} and \tcode{has_service}.

\pnum
In a call to \tcode{use_service<Service>}, the type argument chooses a service, identified by \tcode{Service::key_type}, from a set of services in an \tcode{execution_context}. If the service is not present in the \tcode{execution_context}, an object of type \tcode{Service} is created and added to the \tcode{execution_context}. A program can check if an \tcode{execution_context} implements a particular service with the function template \tcode{has_service<Service>}.

\pnum
Service objects may be explicitly added to an \tcode{execution_context} using the function template \tcode{make_service<Service>}. If the service is already present, \tcode{make_service} exits via an exception of type \tcode{service_already_exists}.

\pnum
Once a service reference is obtained from an \tcode{execution_context} object by calling \tcode{use_service}, that reference remains usable until a call to \tcode{destroy()}.

\pnum
If a call to a specialization of \tcode{use_service} or \tcode{make_service}
recursively calls another specialization of \tcode{use_service} or \tcode{make_service}
which would choose the same service (identified by \tcode{key_type}) from the same \tcode{execution_context},
then the behavior is undefined.
\begin{note}
Nested calls to specializations for different service types are well-defined.
\end{note}

\rSec2[async.exec.ctx.cons]{\tcode{execution_context} constructor}

\indexlibrary{\idxcode{execution_context}!constructor}%
\begin{itemdecl}
execution_context();
\end{itemdecl}

\begin{itemdescr}
\pnum
\effects Creates an object of class \tcode{execution_context} which contains no services. \begin{note} An implementation might preload services of internal service types for its own use. \end{note}
\end{itemdescr}



\rSec2[async.exec.ctx.dtor]{\tcode{execution_context} destructor}

\indexlibrary{\idxcode{execution_context}!destructor}%
\begin{itemdecl}
~execution_context();
\end{itemdecl}

\begin{itemdescr}
\pnum
\effects Destroys an object of class \tcode{execution_context}. Performs \tcode{shutdown()} followed by \tcode{destroy()}.
\end{itemdescr}



\rSec2[async.exec.ctx.ops]{\tcode{execution_context} operations}

\indexlibrarymember{notify_fork}{execution_context}%
\begin{itemdecl}
void notify_fork(fork_event e);
\end{itemdecl}

\begin{itemdescr}
\pnum
\effects For each service object \tcode{svc} in the set:
\begin{itemize}
\item
 If \tcode{e == fork_event::prepare}, performs \tcode{svc->notify_fork(e)} in reverse order of addition to the set.
\item
 Otherwise, performs \tcode{svc->notify_fork(e)} in order of addition to the set.
\end{itemize}
\end{itemdescr}



\rSec2[async.exec.ctx.protected]{\tcode{execution_context} protected operations}

\indexlibrarymember{shutdown}{execution_context}%
\begin{itemdecl}
void shutdown() noexcept;
\end{itemdecl}

\begin{itemdescr}
\pnum
\effects For each service object \tcode{svc} in the \tcode{execution_context} set, in reverse order of addition to the set, performs \tcode{svc->shutdown()}. For each service in the set, \tcode{svc->shutdown()} is called only once irrespective of the number of calls to \tcode{shutdown} on the \tcode{execution_context}.
\end{itemdescr}

\begin{itemdecl}
void destroy() noexcept;
\end{itemdecl}

\begin{itemdescr}
\pnum
\effects Destroys each service object in the \tcode{execution_context} set, and removes it from the set, in reverse order of addition to the set.
\end{itemdescr}



\rSec2[async.exec.ctx.globals]{\tcode{execution_context} globals}

\pnum
The functions \tcode{use_service}, \tcode{make_service}, and \tcode{has_service} do not introduce data races as a result of concurrent calls to those functions from different threads.

\indexlibrary{\idxcode{use_service}}%
\begin{itemdecl}
template<class Service> typename Service::key_type&
  use_service(execution_context& ctx);
\end{itemdecl}

\begin{itemdescr}
\pnum
\effects If an object of type \tcode{Service::key_type} does not already exist in the \tcode{execution_context} set identified by \tcode{ctx}, creates an object of type \tcode{Service}, initialized as \tcode{Service(ctx)}, and adds it to the set.

\pnum
\returns A reference to the corresponding service of \tcode{ctx}.

\pnum
\remarks The reference returned remains valid until a call to \tcode{destroy}.
\end{itemdescr}

\indexlibrary{\idxcode{make_service}}%
\begin{itemdecl}
template<class Service, class... Args> Service&
  make_service(execution_context& ctx, Args&&... args);
\end{itemdecl}

\begin{itemdescr}
\pnum
\requires A service object of type \tcode{Service::key_type} does not already exist in the \tcode{execution_context} set identified by \tcode{ctx}.

\pnum
\effects Creates an object of type \tcode{Service}, initialized as \tcode{Service(ctx, forward<Args>(args)...)}, and adds it to the \tcode{execution_context} set identified by \tcode{ctx}.

\pnum
\throws \tcode{service_already_exists} if a corresponding service object of type \tcode{Service::key_type} is already present in the set.

\pnum
\remarks The reference returned remains valid until a call to \tcode{destroy}.
\end{itemdescr}

\indexlibrary{\idxcode{has_service}}%
\begin{itemdecl}
template<class Service> bool has_service(const execution_context& ctx) noexcept;
\end{itemdecl}

\begin{itemdescr}
\pnum
\returns \tcode{true} if an object of type \tcode{Service::key_type} is present in \tcode{ctx}, otherwise \tcode{false}.
\end{itemdescr}




\rSec1[async.exec.ctx.svc]{Class \tcode{execution_context::service}}

\indexlibrary{\idxcode{execution_context::service}}%
\begin{codeblock}
namespace std {
namespace experimental {
namespace net {
inline namespace v1 {

  class execution_context::service
  {
  protected:
    // construct / copy / destroy:

    explicit service(execution_context& owner);
    service(const service&) = delete;
    service& operator=(const service&) = delete;
    virtual ~service();

    // service observers:

    execution_context& context() noexcept;

  private:
    // service operations:

    virtual void shutdown() noexcept = 0;
    virtual void notify_fork(fork_event e) {}

    execution_context& context_; // \expos
  };

} // inline namespace v1
} // namespace net
} // namespace experimental
} // namespace std
\end{codeblock}

\indexlibrary{\idxcode{execution_context::service}!constructor}%
\begin{itemdecl}
explicit service(execution_context& owner);
\end{itemdecl}

\begin{itemdescr}
\pnum
\postconditions \tcode{std::addressof(context_) == std::addressof(owner)}.
\end{itemdescr}

\indexlibrarymember{context}{execution_context::service}%
\begin{itemdecl}
execution_context& context() noexcept;
\end{itemdecl}

\begin{itemdescr}
\pnum
\returns \tcode{context_}.
\end{itemdescr}



\rSec1[async.executor.arg]{Executor argument tag}

\indexlibrary{\idxcode{executor_arg_t}}%
\indexlibrary{\idxcode{executor_arg}}%
\begin{codeblock}
namespace std {
namespace experimental {
namespace net {
inline namespace v1 {

  struct executor_arg_t { };
  constexpr executor_arg_t executor_arg = executor_arg_t();

} // inline namespace v1
} // namespace net
} // namespace experimental
} // namespace std
\end{codeblock}

\pnum
The \tcode{executor_arg_t} struct is an empty structure type used as a unique type to disambiguate constructor and function overloading. Specifically, types may have constructors with \tcode{executor_arg_t} as the first argument, immediately followed by an argument of a type that satisfies the general requirements on executors (P0443R5).



\rSec1[async.uses.executor]{\tcode{uses_executor}}


\rSec2[async.uses.executor.trait]{\tcode{uses_executor} trait}

\indexlibrary{\idxcode{uses_executor}}%
\begin{codeblock}
namespace std {
namespace experimental {
namespace net {
inline namespace v1 {

  template<class T, class Executor> struct uses_executor;

} // inline namespace v1
} // namespace net
} // namespace experimental
} // namespace std
\end{codeblock}

\pnum
Remark: Detects whether \tcode{T} has a nested \tcode{executor_type} that is convertible from \tcode{Executor}. Meets the \tcode{BinaryTypeTrait} requirements (\CppXref{meta.rqmts}). The implementation provides a definition that is derived from \tcode{true_type} if a type \tcode{T::executor_type} exists and \tcode{is_convertible<Executor, T::executor_type>::value != false}, otherwise it is derived from \tcode{false_type}. A program may specialize this template to derive from \tcode{true_type} for a user-defined type \tcode{T} that does not have a nested \tcode{executor_type} but nonetheless can be constructed with an executor if the first argument of a constructor has type \tcode{executor_arg_t} and the second argument has type \tcode{Executor}.



\rSec2[async.uses.executor.cons]{uses-executor construction}

\indextext{uses-executor construction}%
\pnum
Uses-executor construction with executor \tcode{Executor} refers to the construction of an object \tcode{obj} of type \tcode{T}, using constructor arguments \tcode{v1, v2, ..., vN} of types \tcode{V1, V2, ..., VN}, respectively, and an executor \tcode{ex} of type \tcode{Executor}, according to the following rules:

\begin{itemize}
\item
if \tcode{uses_executor_v<T, Executor>} is \tcode{true}
and \tcode{is_constructible<T, executor_arg_t, Executor, V1, V2, ..., VN>::value} is \tcode{true},
then \tcode{obj} is initialized as \tcode{obj(executor_arg, ex, v1, v2, ..., vN)};

\item
otherwise, \tcode{obj} is initialized as \tcode{obj(v1, v2, ..., vN)}.
\end{itemize}




\rSec1[async.assoc.exec]{Class template \tcode{associated_executor}}

\indexlibrary{\idxcode{associated_executor}}%
\pnum
Class template \tcode{associated_executor} is an associator~(\ref{async.reqmts.associator}) for executors, with default candidate type \tcode{system_executor} and default candidate object \tcode{system_executor()}.

\begin{codeblock}
namespace std {
namespace experimental {
namespace net {
inline namespace v1 {

  template<class T, class Executor = system_executor>
  struct associated_executor
  {
    using type = @\seebelow@;

    static type get(const T& t, const Executor& e = Executor()) noexcept;
  };

} // inline namespace v1
} // namespace net
} // namespace experimental
} // namespace std
\end{codeblock}

\indextext{requirements!\idxcode{associated_executor} specializations}%
\pnum
Specializations of \tcode{associated_executor} shall satisfy the requirements in Table~\ref{tab:async.assoc.exec.requirements}.
In this table, \tcode{X} is a specialization of \tcode{associated_executor}
for the template parameters \tcode{T} and \tcode{Executor};
\tcode{t} is a (possible const) value of \tcode{T};
and \tcode{e} is an object of type \tcode{Executor}.

\begin{libreqtab3}
{\tcode{associated_executor} specialization requirements}
{tab:async.assoc.exec.requirements}
\\ \topline
\lhdr{Expression}  &
\chdr{Return type}  &
\rhdr{Note} \\ \capsep
\endfirsthead
\continuedcaption\\
\hline
\lhdr{Expression}  &
\chdr{Return type}  &
\rhdr{Note} \\ \capsep
\endhead

\tcode{typename X::type}  &
A type satisfying the general requirements on executors (P0443R5).  &
  \\ \rowsep

\tcode{X::get(t)}  &
\tcode{X::type}  &
Shall not exit via an exception. Equivalent to \tcode{X::get(t, Executor())}.  \\ \rowsep

\tcode{X::get(t, e)}  &
\tcode{X::type}  &
 Shall not exit via an exception.  \\

\end{libreqtab3}


\rSec2[async.assoc.exec.members]{\tcode{associated_executor} members}

\begin{itemdecl}
using type = @\seebelow@;
\end{itemdecl}

\begin{itemdescr}
\pnum
\ctype \tcode{T::executor_type} if the \grammarterm{qualified-id} \tcode{T::executor_type}
is valid and denotes a type~(\CppXref{temp.deduct}).
Otherwise \tcode{Executor}.
\end{itemdescr}

\begin{itemdecl}
type get(const T& t, const Executor& e = Executor()) noexcept;
\end{itemdecl}

\begin{itemdescr}
\pnum
\returns \tcode{t.get_executor()} if the \grammarterm{qualified-id} \tcode{T::executor_type}
is valid and denotes a type~(\CppXref{temp.deduct}).
Otherwise \tcode{e}.
\end{itemdescr}




\rSec1[async.assoc.exec.get]{Function \tcode{get_associated_executor}}

\indexlibrary{\idxcode{get_associated_executor}}%
\begin{itemdecl}
template<class T>
  associated_executor_t<T> get_associated_executor(const T& t) noexcept;
\end{itemdecl}

\begin{itemdescr}
\pnum
\returns \tcode{associated_executor<T>::get(t)}.
\end{itemdescr}

\begin{itemdecl}
template<class T, class Executor>
  associated_executor_t<T, Executor>
    get_associated_executor(const T& t, const Executor& ex) noexcept;
\end{itemdecl}

\begin{itemdescr}
\pnum
\returns \tcode{associated_executor<T, Executor>::get(t, ex)}.

\pnum
\remarks This function shall not participate in overload resolution unless
\tcode{is_convertible<Executor\&, execution_context\&>::value} is \tcode{false}.
\end{itemdescr}

\begin{itemdecl}
template<class T, class ExecutionContext>
  associated_executor_t<T, typename ExecutionContext::executor_type>
    get_associated_executor(const T& t, ExecutionContext& ctx) noexcept;
\end{itemdecl}

\begin{itemdescr}
\pnum
\returns \tcode{get_associated_executor(t, ctx.get_executor())}.

\pnum
\remarks This function shall not participate in overload resolution unless
\tcode{is_convertible<Exec\-ution\-Context\&, execution_context\&>::value} is \tcode{true}.
\end{itemdescr}



\rSec1[async.exec.binder]{Class template \tcode{executor_binder}}

\indexlibrary{\idxcode{executor_binder}}%
\pnum
The class template \tcode{executor_binder} binds executors to objects.
A specialization \tcode{executor_binder<T, Executor>} binds
an executor of type \tcode{Executor} satisfying the general requirements on executors (P0443R5)
to an object or function of type \tcode{T}.

\begin{codeblock}
namespace std {
namespace experimental {
namespace net {
inline namespace v1 {

  template<class T, class Executor>
  class executor_binder
  {
  public:
    // types:

    using target_type = T;
    using executor_type = Executor;

    // \ref{async.exec.binder.cons}, construct / copy / destroy:

    executor_binder(T t, const Executor& ex);
    executor_binder(const executor_binder& other) = default;
    executor_binder(executor_binder&& other) = default;
    template<class U, class OtherExecutor>
      executor_binder(const executor_binder<U, OtherExecutor>& other);
    template<class U, class OtherExecutor>
      executor_binder(executor_binder<U, OtherExecutor>&& other);
    template<class U, class OtherExecutor>
      executor_binder(executor_arg_t, const Executor& ex,
        const executor_binder<U, OtherExecutor>& other);
    template<class U, class OtherExecutor>
      executor_binder(executor_arg_t, const Executor& ex,
        executor_binder<U, OtherExecutor>&& other);

    ~executor_binder();

    // \ref{async.exec.binder.access}, executor binder access:

    T& get() noexcept;
    const T& get() const noexcept;
    executor_type get_executor() const noexcept;

    // \ref{async.exec.binder.invocation}, executor binder invocation:

    template<class... Args>
      result_of_t<T&(Args&&...)> operator()(Args&&... args);
    template<class... Args>
      result_of_t<const T&(Args&&...)> operator()(Args&&... args) const;

  private:
    Executor ex_; // \expos
    T target_; // \expos
  };

  template<class T, class Executor, class Signature>
    class async_result<executor_binder<T, Executor>, Signature>;

  template<class T, class Executor, class ProtoAllocator>
    struct associated_allocator<executor_binder<T, Executor>, ProtoAllocator>;

  template<class T, class Executor, class Executor1>
    struct associated_executor<executor_binder<T, Executor>, Executor1>;

} // inline namespace v1
} // namespace net
} // namespace experimental
} // namespace std
\end{codeblock}


\rSec2[async.exec.binder.cons]{\tcode{executor_binder} constructors}

\indexlibrary{\idxcode{executor_binder}!constructor}%
\begin{itemdecl}
executor_binder(T t, const Executor& ex);
\end{itemdecl}

\begin{itemdescr}
\pnum
\effects Initializes \tcode{ex_} with \tcode{ex}. Initializes \tcode{target_} by performing uses-executor construction, using the constructor argument \tcode{std::move(t)} and the executor \tcode{ex_}.
\end{itemdescr}

\begin{itemdecl}
template<class U, class OtherExecutor>
  executor_binder(const executor_binder<U, OtherExecutor>& other);
\end{itemdecl}

\begin{itemdescr}
\pnum
\requires If \tcode{U} is not convertible to \tcode{T}, or if \tcode{OtherExecutor} is not convertible to \tcode{Executor}, the program is ill-formed.

\pnum
\effects Initializes \tcode{ex_} with \tcode{other.get_executor()}. Initializes \tcode{target_} by performing uses-executor construction, using the constructor argument \tcode{other.get()} and the executor \tcode{ex_}.
\end{itemdescr}

\begin{itemdecl}
template<class U, class OtherExecutor>
  executor_binder(executor_binder<U, OtherExecutor>&& other);
\end{itemdecl}

\begin{itemdescr}
\pnum
\requires If \tcode{U} is not convertible to \tcode{T}, or if \tcode{OtherExecutor} is not convertible to \tcode{Executor}, the program is ill-formed.

\pnum
\effects Initializes \tcode{ex_} with \tcode{other.get_executor()}. Initializes \tcode{target_} by performing uses-executor construction, using the constructor argument \tcode{std::move(other.get())} and the executor \tcode{ex_}.
\end{itemdescr}

\begin{itemdecl}
template<class U, class OtherExecutor>
  executor_binder(executor_arg_t, const Executor& ex,
    const executor_binder<U, OtherExecutor>& other);
\end{itemdecl}

\begin{itemdescr}
\pnum
\requires If \tcode{U} is not convertible to \tcode{T} the program is ill-formed.

\pnum
\effects Initializes \tcode{ex_} with \tcode{ex}. Initializes \tcode{target_} by performing uses-executor construction, using the constructor argument \tcode{other.get()} and the executor \tcode{ex_}.
\end{itemdescr}

\begin{itemdecl}
template<class U, class OtherExecutor>
  executor_binder(executor_arg_t, const Executor& ex,
    executor_binder<U, OtherExecutor>&& other);
\end{itemdecl}

\begin{itemdescr}
\pnum
\requires \tcode{U} is \tcode{T} or convertible to \tcode{T}.

\pnum
\effects Initializes \tcode{ex_} with \tcode{ex}. Initializes \tcode{target_} by performing uses-executor construction, using the constructor argument \tcode{std::move(other.get())} and the executor \tcode{ex_}.
\end{itemdescr}



\rSec2[async.exec.binder.access]{\tcode{executor_binder} access}

\indexlibrarymember{get}{executor_binder}%
\begin{itemdecl}
T& get() noexcept;
const T& get() const noexcept;
\end{itemdecl}

\begin{itemdescr}
\pnum
\returns \tcode{target_}.
\end{itemdescr}

\indexlibrarymember{get_executor}{executor_binder}%
\begin{itemdecl}
executor_type get_executor() const noexcept;
\end{itemdecl}

\begin{itemdescr}
\pnum
\returns \tcode{executor_}.
\end{itemdescr}



\rSec2[async.exec.binder.invocation]{\tcode{executor_binder} invocation}

\indexlibrarymember{operator()}{executor_binder}%
\begin{itemdecl}
template<class... Args>
  result_of_t<T&(Args&&...)> operator()(Args&&... args);
template<class... Args>
  result_of_t<const T&(Args&&...)> operator()(Args&&... args) const;
\end{itemdecl}

\begin{itemdescr}
\pnum
\returns \tcode{\placeholdernc{INVOKE}(get(), forward<Args>(args)...)} (\CppXref{func.require}).
\end{itemdescr}



\rSec2[async.exec.binder.async.result]{Class template partial specialization \tcode{async_result}}

\indextext{\idxcode{async_result}!specialization for \tcode{executor_binder}}%
\indexlibrary{\idxcode{async_result}}%
\begin{codeblock}
namespace std {
namespace experimental {
namespace net {
inline namespace v1 {

  template<class T, class Executor, class Signature>
  class async_result<executor_binder<T, Executor>, Signature>
  {
  public:
    using completion_handler_type = executor_binder<
      typename async_result<T, Signature>::completion_handler_type,
        Executor>;
    using return_type = typename async_result<T, Signature>::return_type;

    explicit async_result(completion_handler_type& h);
    async_result(const async_result&) = delete;
    async_result& operator=(const async_result&) = delete;

    return_type get();

  private:
    async_result<T, Signature> target_; // \expos
  };

} // inline namespace v1
} // namespace net
} // namespace experimental
} // namespace std
\end{codeblock}

\begin{itemdecl}
explicit async_result(completion_handler_type& h);
\end{itemdecl}

\begin{itemdescr}
\pnum
\effects Initializes \tcode{target_} as \tcode{target_(h.get())}.
\end{itemdescr}

\begin{itemdecl}
return_type get();
\end{itemdecl}

\begin{itemdescr}
\pnum
\returns \tcode{target_.get()}.
\end{itemdescr}



\rSec2[async.exec.binder.assoc.alloc]{Class template partial specialization \tcode{associated_allocator}}

\indextext{\idxcode{associated_allocator}!specialization for \tcode{executor_binder}}%
\indexlibrary{\idxcode{associated_allocator}}%
\begin{codeblock}
namespace std {
namespace experimental {
namespace net {
inline namespace v1 {

  template<class T, class Executor, class ProtoAllocator>
    struct associated_allocator<executor_binder<T, Executor>, ProtoAllocator>
  {
    using type = associated_allocator_t<T, ProtoAllocator>;

    static type get(const executor_binder<T, Executor>& b,
                    const ProtoAllocator& a = ProtoAllocator()) noexcept;
  };

} // inline namespace v1
} // namespace net
} // namespace experimental
} // namespace std
\end{codeblock}

\begin{itemdecl}
static type get(const executor_binder<T, Executor>& b,
                const ProtoAllocator& a = ProtoAllocator()) noexcept;
\end{itemdecl}

\begin{itemdescr}
\pnum
\returns \tcode{associated_allocator<T, ProtoAllocator>::get(b.get(), a)}.
\end{itemdescr}



\rSec2[async.exec.binder.assoc.exec]{Class template partial specialization \tcode{associated_executor}}

\indextext{\idxcode{associated_executor}!specialization for \tcode{executor_binder}}%
\indexlibrary{\idxcode{associated_executor}}%
\begin{codeblock}
namespace std {
namespace experimental {
namespace net {
inline namespace v1 {

  template<class T, class Executor, class Executor1>
    struct associated_executor<executor_binder<T, Executor>, Executor1>
  {
    using type = Executor;

    static type get(const executor_binder<T, Executor>& b,
                    const Executor1& e = Executor1()) noexcept;
  };

} // inline namespace v1
} // namespace net
} // namespace experimental
} // namespace std
\end{codeblock}

\begin{itemdecl}
static type get(const executor_binder<T, Executor>& b,
                const Executor1& e = Executor1()) noexcept;
\end{itemdecl}

\begin{itemdescr}
\pnum
\returns \tcode{b.get_executor()}.
\end{itemdescr}




\rSec1[async.bind.executor]{Function \tcode{bind_executor}}

\indexlibrary{\idxcode{bind_executor}}%
\begin{itemdecl}
template<class Executor, class T>
  executor_binder<decay_t<T>, Executor>
    bind_executor(const Executor& ex, T&& t);
\end{itemdecl}

\begin{itemdescr}
\pnum
\returns \tcode{executor_binder<decay_t<T>, Executor>(forward<T>(t), ex)}.

\pnum
\remarks This function shall not participate in overload resolution unless
\tcode{is_convertible<Executor\&, execution_context\&>::value} is \tcode{false}.
\end{itemdescr}

\indexlibrary{\idxcode{bind_executor}}%
\begin{itemdecl}
template<class ExecutionContext, class CompletionToken>
  executor_binder<decay_t<T>, typename ExecutionContext::executor_type>
    bind_executor(ExecutionContext& ctx, T&& t);
\end{itemdecl}

\begin{itemdescr}
\pnum
\returns \tcode{bind_executor(ctx.get_executor(), forward<T>(t))}.

\pnum
\remarks This function shall not participate in overload resolution unless
\tcode{is_convertible<Exec\-ution\-Context\&, execution_context\&>::value} is \tcode{true}.
\end{itemdescr}



\rSec1[async.system.context]{Class \tcode{system_context}}

\indexlibrary{\idxcode{system_context}}%
\pnum
Class \tcode{system_context} implements an execution context that represents the ability to run a submitted function object on any thread.

\begin{codeblock}
namespace std {
namespace experimental {
namespace net {
inline namespace v1 {

  class system_context : public execution_context
  {
  public:
    // types:

    using executor_type = @\seebelow@;

    // construct / copy / destroy:

    system_context() = delete;
    system_context(const system_context&) = delete;
    system_context& operator=(const system_context&) = delete;
    ~system_context();

    // system_context operations:

    executor_type get_executor() noexcept;

    void stop();
    bool stopped() const noexcept;
    void join();
  };

} // inline namespace v1
} // namespace net
} // namespace experimental
} // namespace std
\end{codeblock}

\pnum
The class \tcode{system_context} satisfies the \tcode{ExecutionContext}~(\ref{async.reqmts.executioncontext}) type requirements.

\indexlibrarymember{executor_type}{system_context}%
\pnum
\tcode{executor_type} is an executor type conforming to the specification for \tcode{system_context} executor types described below. Executor objects of type \tcode{executor_type} have the following properties established:
\begin{itemize}
\item
\tcode{execution::oneway}
\item
\tcode{execution::single}
\item
\tcode{execution::possibly_blocking}
\item
\tcode{execution::not_continuation}
\item
\tcode{execution::thread_execution_mapping}
\item
\tcode{execution::allocator(std::allocator<void>())}
\end{itemize}

\pnum
\tcode{system_context} executors having a different set of established properties are represented by a distinct, unspecified type. Function objects submitted via a \tcode{system_context} executor object are permitted to execute on any thread. To satisfy the requirements for the \tcode{execution::never_blocking} property, a \tcode{system_context} executor may create \tcode{thread} objects to run the submitted function objects. These \tcode{thread} objects are collectively referred to as system threads.

\pnum
The \tcode{system_context} member functions \tcode{get_executor}, \tcode{stop}, and \tcode{stopped}, and the \tcode{system_executor} copy constructors, member functions and comparison operators, do not introduce data races as a result of concurrent calls to those functions from different threads of execution.

\indexlibrary{\idxcode{system_context}!destructor}%
\begin{itemdecl}
~system_context();
\end{itemdecl}

\begin{itemdescr}
\pnum
\effects Performs \tcode{stop()} followed by \tcode{join()}.
\end{itemdescr}

\indexlibrarymember{get_executor}{system_context}%
\begin{itemdecl}
executor_type get_executor() noexcept;
\end{itemdecl}

\begin{itemdescr}
\pnum
\returns \tcode{executor_type()}.
\end{itemdescr}

\indexlibrarymember{stop}{system_context}%
\begin{itemdecl}
void stop();
\end{itemdecl}

\begin{itemdescr}
\pnum
\effects Signals all system threads to exit as soon as possible. If a system thread is currently executing a function object, the thread will exit only after completion of that function object. Returns without waiting for the system threads to complete.

\pnum
\postconditions \tcode{stopped() == true}.
\end{itemdescr}

\indexlibrarymember{stopped}{system_context}%
\begin{itemdecl}
bool stopped() const noexcept;
\end{itemdecl}

\begin{itemdescr}
\pnum
\returns \tcode{true} if the \tcode{system_context} has been stopped by a prior call to \tcode{stop}.
\end{itemdescr}

\indexlibrarymember{join}{system_context}%
\begin{itemdecl}
void join();
\end{itemdecl}

\begin{itemdescr}
\pnum
\effects Blocks the calling thread (\CppXref{defns.block}) until all system threads have completed.

\pnum
\sync The completion of each system thread synchronizes with (\CppXref{intro.multithread}) the corresponding successful \tcode{join()} return.
\end{itemdescr}



\rSec1[async.system.exec]{\tcode{system_context} executor types}

\pnum
All executor types accessible through \tcode{system_context::executor_type()}, \tcode{system_context::get_executor()}, and subsequent calls to the member function \tcode{require}, conform to the following specification.

\indexlibrary{\idxcode{system_context::executor_type}}%
\begin{codeblock}
namespace std {
namespace experimental {
namespace net {
inline namespace v1 {

  class @\placeholder{C}@
  {
  public:
    // construct / copy / destroy:

    @\placeholder{C}@() {}

    // \ref{async.system.exec.ops}, executor operations:

    @\seebelow@ require(execution::never_blocking_t) const;
    @\seebelow@ require(execution::possibly_blocking_t) const;
    @\seebelow@ require(execution::continuation_t) const;
    @\seebelow@ require(execution::not_continuation_t) const;
    @\seebelow@ require(execution::allocator_t<void>) const;
    template<class ProtoAllocator>
      @\seebelow@ require(const execution::allocator_t<ProtoAllocator>& a) const;

    static constexpr bool query(execution::thread_execution_mapping_t) noexcept;
    static system_context& query(execution::context_t) noexcept;
    @\seebelow@ query(execution::allocator_t<void>) const noexcept;
    template<class ProtoAllocator>
      @\seebelow@ query(const execution::allocator_t<ProtoAllocator>&) const noexcept;

    template<class Function>
      void execute(Function&& f) const;
  };

  bool operator==(const @\placeholder{C}@& a, const @\placeholder{C}@& b) noexcept;
  bool operator!=(const @\placeholder{C}@& a, const @\placeholder{C}@& b) noexcept;

} // inline namespace v1
} // namespace net
} // namespace experimental
} // namespace std
\end{codeblock}

\pnum
\placeholder{C} is a type satisfying the OneWayExecutor requirements~(P0443R5).

\rSec2[async.system.exec.ops]{\tcode{system_context} executor operations}

\begin{itemdecl}
@\seebelow@ require(execution::never_blocking_t) const;
@\seebelow@ require(execution::possibly_blocking_t) const;
@\seebelow@ require(execution::continuation_t) const;
@\seebelow@ require(execution::not_continuation_t) const;
\end{itemdecl}

\begin{itemdescr}
\pnum
\returns A \tcode{system_context} executor object of an unspecified type conforming to these specifications, with the requested property established. When the requested property is part of a group that is defined as a mutually exclusive set, any other properties in the group are removed from the returned executor object. All other properties of the returned executor object are identical to those of \tcode{*this}.
\end{itemdescr}

\begin{itemdecl}
@\seebelow@ require(execution::allocator_t<void>) const;
\end{itemdecl}

\begin{itemdescr}
\pnum
\returns \tcode{require(execution::allocator(std::allocator<void>()))}.
\end{itemdescr}

\begin{itemdecl}
template<class ProtoAllocator>
  @\seebelow@ require(const execution::allocator_t<ProtoAllocator>& a) const;
\end{itemdecl}

\begin{itemdescr}
\pnum
\returns A \tcode{system_context} executor object of an unspecified type conforming to these specifications, with the \tcode{execution::allocator_t<ProtoAllocator>} property established such that allocation and deallocation associated with function submission will be performed using a copy of \tcode{a.value()}. All other properties of the returned executor object are identical to those of \tcode{*this}.
\end{itemdescr}

\begin{itemdecl}
static constexpr bool query(execution::thread_execution_mapping_t) noexcept;
\end{itemdecl}

\begin{itemdescr}
\pnum
\returns \tcode{true}.
\end{itemdescr}

\begin{itemdecl}
static system_context& query(execution::context_t) const;
\end{itemdecl}

\begin{itemdescr}
\pnum
\returns A reference to the \tcode{system_context} object.
\end{itemdescr}

\begin{itemdecl}
@\seebelow@ query(execution::allocator_t<void>) const noexcept;
template<class ProtoAllocator>
  @\seebelow@ query(const execution::allocator_t<ProtoAllocator>&) const noexcept;
\end{itemdecl}

\begin{itemdescr}
\pnum
\returns The allocator object associated with the executor, with type and value as previously established by the \tcode{execution::allocator_t<ProtoAllocator>} property.
\end{itemdescr}

\indexlibrarymember{execute}{system_context::executor_type}%
\begin{itemdecl}
template<class Function>
  void execute(Function&& f) const
\end{itemdecl}

\begin{itemdescr}
\pnum
\effects Submits the function \tcode{f} for execution according to the OneWayExecutor requirements and the properties established for \tcode{*this}. If \tcode{f} exits via an exception, calls \tcode{std::terminate()}.
\end{itemdescr}



\rSec2[async.system.exec.comparisons]{\tcode{system_context} executor comparisons}

\indexlibrarymember{operator==}{system_context::executor_type}%
\begin{itemdecl}
bool operator==(const @\placeholder{C}@& a, const @\placeholder{C}@& b) noexcept;
\end{itemdecl}

\begin{itemdescr}
\pnum
\returns \tcode{true}.
\end{itemdescr}

\indexlibrarymember{operator"!=}{system_context::executor_type}%
\begin{itemdecl}
bool operator!=(const @\placeholder{C}@& a, const @\placeholder{C}@& b) noexcept;
\end{itemdecl}

\begin{itemdescr}
\pnum
\returns \tcode{false}.
\end{itemdescr}




\rSec1[async.bad.exec]{Class \tcode{bad_executor}}

\indexlibrary{\idxcode{bad_executor}}%
\pnum
An exception of type \tcode{bad_executor} is thrown by \tcode{executor} member functions \tcode{dispatch}, \tcode{post}, and \tcode{defer} when the executor object has no target.

\begin{codeblock}
namespace std {
namespace experimental {
namespace net {
inline namespace v1 {

  class bad_executor : public exception
  {
  public:
    // constructor:
    bad_executor() noexcept;
  };

} // inline namespace v1
} // namespace net
} // namespace experimental
} // namespace std
\end{codeblock}

\begin{itemdecl}
bad_executor() noexcept;
\end{itemdecl}

\begin{itemdescr}
\pnum
\effects constructs a \tcode{bad_executor} object.

\pnum
\postconditions
\tcode{what()} returns an \impldef{result of \tcode{bad_executor::what}} \ntbs.
\end{itemdescr}



\rSec1[async.executor]{Class \tcode{executor}}

\indexlibrary{\idxcode{executor}}%
\pnum
The \tcode{executor} class provides a polymorphic wrapper for types that satisfy the Executor requirements~(\ref{async.reqmts.executor}).

\begin{codeblock}
namespace std {
namespace experimental {
namespace net {
inline namespace v1 {

  class executor
  {
  public:
    // \ref{async.executor.cons}, construct / copy / destroy:

    executor() noexcept;
    executor(nullptr_t) noexcept;
    executor(const executor& e) noexcept;
    executor(executor&& e) noexcept;
    template<class Executor> executor(Executor e);
    template<class Executor, class ProtoAllocator>
      executor(allocator_arg_t, const ProtoAllocator& a, Executor e);

    executor& operator=(const executor& e) noexcept;
    executor& operator=(executor&& e) noexcept;
    executor& operator=(nullptr_t) noexcept;
    template<class Executor> executor& operator=(Executor e);

    ~executor();

    // \ref{async.executor.modifiers}, executor modifiers:

    void swap(executor& other) noexcept;
    template<class Executor, class ProtoAllocator>
      void assign(Executor e, const ProtoAllocator& a);

    // \ref{async.executor.ops}, executor operations:

    execution_context& context() const noexcept;

    void on_work_started() const noexcept;
    void on_work_finished() const noexcept;

    template<class Func, class ProtoAllocator>
      void dispatch(Func&& f, const ProtoAllocator& a) const;
    template<class Func, class ProtoAllocator>
      void post(Func&& f, const ProtoAllocator& a) const;
    template<class Func, class ProtoAllocator>
      void defer(Func&& f, const ProtoAllocator& a) const;

    // \ref{async.executor.capacity}, executor capacity:

    explicit operator bool() const noexcept;

    // \ref{async.executor.target}, executor target access:

    const type_info& target_type() const noexcept;
    template<class Executor> Executor* target() noexcept;
    template<class Executor> const Executor* target() const noexcept;
  };

  // \ref{async.executor.comparisons}, executor comparisons:

  bool operator==(const executor& a, const executor& b) noexcept;
  bool operator==(const executor& e, nullptr_t) noexcept;
  bool operator==(nullptr_t, const executor& e) noexcept;
  bool operator!=(const executor& a, const executor& b) noexcept;
  bool operator!=(const executor& e, nullptr_t) noexcept;
  bool operator!=(nullptr_t, const executor& e) noexcept;

  // \ref{async.executor.algo}, executor specialized algorithms:

  void swap(executor& a, executor& b) noexcept;

} // inline namespace v1
} // namespace net
} // namespace experimental

  template<class Allocator>
    struct uses_allocator<experimental::net::v1::executor, Allocator>
      : true_type {};

} // namespace std
\end{codeblock}

\pnum
Class \tcode{executor} meets the requirements of \tcode{Executor}~(\ref{async.reqmts.executor}), \tcode{DefaultConstructible} (\CppXref{defaultconstructible}), and \tcode{CopyAssignable} (\CppXref{copyassignable}).

\pnum
\begin{note} To meet the \tcode{noexcept} requirements for executor copy constructors and move constructors, implementations may share a target between two or more \tcode{executor} objects. \end{note}

\pnum
The \defnx{target}{target!\tcode{executor}} is the executor object that is held by the wrapper.


\rSec2[async.executor.cons]{\tcode{executor} constructors}

\indexlibrary{\idxcode{executor}!constructor}%
\begin{itemdecl}
executor() noexcept;
\end{itemdecl}

\begin{itemdescr}
\pnum
\postconditions \tcode{!*this}.
\end{itemdescr}

\indexlibrary{\idxcode{executor}!constructor}%
\begin{itemdecl}
executor(nullptr_t) noexcept;
\end{itemdecl}

\begin{itemdescr}
\pnum
\postconditions \tcode{!*this}.
\end{itemdescr}

\indexlibrary{\idxcode{executor}!constructor}%
\begin{itemdecl}
executor(const executor& e) noexcept;
\end{itemdecl}

\begin{itemdescr}
\pnum
\postconditions \tcode{!*this} if \tcode{!e}; otherwise, \tcode{*this} targets \tcode{e.target()} or a copy of \tcode{e.target()}.
\end{itemdescr}

\indexlibrary{\idxcode{executor}!constructor}%
\begin{itemdecl}
executor(executor&& e) noexcept;
\end{itemdecl}

\begin{itemdescr}
\pnum
\effects If \tcode{!e}, \tcode{*this} has no target; otherwise, moves \tcode{e.target()} or move-constructs the target of \tcode{e} into the target of \tcode{*this}, leaving \tcode{e} in a valid state with an unspecified value.
\end{itemdescr}

\indexlibrary{\idxcode{executor}!constructor}%
\begin{itemdecl}
template<class Executor> executor(Executor e);
\end{itemdecl}

\begin{itemdescr}
\pnum
\effects \tcode{*this} targets a copy of \tcode{e} initialized with \tcode{std::move(e)}.
\end{itemdescr}

\indexlibrary{\idxcode{executor}!constructor}%
\begin{itemdecl}
template<class Executor, class ProtoAllocator>
  executor(allocator_arg_t, const ProtoAllocator& a, Executor e);
\end{itemdecl}

\begin{itemdescr}
\pnum
\effects \tcode{*this} targets a copy of \tcode{e} initialized with \tcode{std::move(e)}.

\pnum
A copy of the allocator argument is used to allocate memory, if necessary, for the internal data structures of the constructed \tcode{executor} object.
\end{itemdescr}



\rSec2[async.executor.assign]{\tcode{executor} assignment}

\indexlibrarymember{operator=}{executor}%
\begin{itemdecl}
executor& operator=(const executor& e) noexcept;
\end{itemdecl}

\begin{itemdescr}
\pnum
\effects \tcode{executor(e).swap(*this)}.

\pnum
\returns \tcode{*this}.
\end{itemdescr}

\indexlibrarymember{operator=}{executor}%
\begin{itemdecl}
executor& operator=(executor&& e) noexcept;
\end{itemdecl}

\begin{itemdescr}
\pnum
\effects Replaces the target of \tcode{*this} with the target of \tcode{e}, leaving \tcode{e} in a valid state with an unspecified value.

\pnum
\returns \tcode{*this}.
\end{itemdescr}

\indexlibrarymember{operator=}{executor}%
\begin{itemdecl}
executor& operator=(nullptr_t) noexcept;
\end{itemdecl}

\begin{itemdescr}
\pnum
\effects \tcode{executor(nullptr).swap(*this)}.

\pnum
\returns \tcode{*this}.
\end{itemdescr}

\indexlibrarymember{operator=}{executor}%
\begin{itemdecl}
template<class Executor> executor& operator=(Executor e);
\end{itemdecl}

\begin{itemdescr}
\pnum
\effects \tcode{executor(std::move(e)).swap(*this)}.

\pnum
\returns \tcode{*this}.
\end{itemdescr}



\rSec2[async.executor.dtor]{\tcode{executor} destructor}

\indexlibrary{\idxcode{executor}!destructor}%
\begin{itemdecl}
~executor();
\end{itemdecl}

\begin{itemdescr}
\pnum
\effects If \tcode{*this != nullptr}, releases shared ownership of, or destroys, the target of \tcode{*this}.
\end{itemdescr}



\rSec2[async.executor.modifiers]{\tcode{executor} modifiers}

\indexlibrarymember{swap}{executor}%
\begin{itemdecl}
void swap(executor& other) noexcept;
\end{itemdecl}

\begin{itemdescr}
\pnum
\effects Interchanges the targets of \tcode{*this} and \tcode{other}.
\end{itemdescr}

\indexlibrarymember{assign}{executor}%
\begin{itemdecl}
template<class Executor, class ProtoAllocator>
  void assign(Executor e, const ProtoAllocator& a);
\end{itemdecl}

\begin{itemdescr}
\pnum
\effects \tcode{executor(allocator_arg, a, std::move(e)).swap(*this)}.
\end{itemdescr}



\rSec2[async.executor.ops]{\tcode{executor} operations}

\indexlibrarymember{context}{executor}%
\begin{itemdecl}
execution_context& context() const noexcept;
\end{itemdecl}

\begin{itemdescr}
\pnum
\requires \tcode{*this != nullptr}.

\pnum
\returns \tcode{e.context()}, where \tcode{e} is the target object of \tcode{*this}.
\end{itemdescr}

\indexlibrarymember{on_work_started}{executor}%
\begin{itemdecl}
void on_work_started() const noexcept;
\end{itemdecl}

\begin{itemdescr}
\pnum
\requires \tcode{*this != nullptr}.

\pnum
\effects \tcode{e.on_work_started()}, where \tcode{e} is the target object of \tcode{*this}.
\end{itemdescr}

\indexlibrarymember{on_work_finished}{executor}%
\begin{itemdecl}
void on_work_finished() const noexcept;
\end{itemdecl}

\begin{itemdescr}
\pnum
\requires \tcode{*this != nullptr}.

\pnum
\effects \tcode{e.on_work_finished()}, where \tcode{e} is the target object of \tcode{*this}.
\end{itemdescr}

\indexlibrarymember{dispatch}{executor}%
\begin{itemdecl}
template<class Func, class ProtoAllocator>
  void dispatch(Func&& f, const ProtoAllocator& a) const;
\end{itemdecl}

\begin{itemdescr}
\pnum
Let \tcode{e} be the target object of \tcode{*this}. Let \tcode{a1} be the allocator that was specified when the target was set. Let \tcode{fd} be the result of \tcode{\placeholdernc{DECAY_COPY}(f)} (\CppXref{thread.decaycopy}).

\pnum
\effects \tcode{e.dispatch(g, a1)}, where \tcode{g} is a function object of unspecified type that, when called as \tcode{g()}, performs \tcode{fd()}. The allocator \tcode{a} is used to allocate any memory required to implement \tcode{g}.
\end{itemdescr}

\indexlibrarymember{post}{executor}%
\begin{itemdecl}
template<class Func, class ProtoAllocator>
  void post(Func&& f, const ProtoAllocator& a) const;
\end{itemdecl}

\begin{itemdescr}
\pnum
Let \tcode{e} be the target object of \tcode{*this}. Let \tcode{a1} be the allocator that was specified when the target was set. Let \tcode{fd} be the result of \tcode{\placeholdernc{DECAY_COPY}(f)}.

\pnum
\effects \tcode{e.post(g, a1)}, where \tcode{g} is a function object of unspecified type that, when called as \tcode{g()}, performs \tcode{fd()}. The allocator \tcode{a} is used to allocate any memory required to implement \tcode{g}.
\end{itemdescr}

\indexlibrarymember{defer}{executor}%
\begin{itemdecl}
template<class Func, class ProtoAllocator>
  void defer(Func&& f, const ProtoAllocator& a) const;
\end{itemdecl}

\begin{itemdescr}
\pnum
Let \tcode{e} be the target object of \tcode{*this}. Let \tcode{a1} be the allocator that was specified when the target was set. Let \tcode{fd} be the result of \tcode{\placeholdernc{DECAY_COPY}(f)}.

\pnum
\effects \tcode{e.defer(g, a1)}, where \tcode{g} is a function object of unspecified type that, when called as \tcode{g()}, performs \tcode{fd()}. The allocator \tcode{a} is used to allocate any memory required to implement \tcode{g}.
\end{itemdescr}



\rSec2[async.executor.capacity]{\tcode{executor} capacity}

\indexlibrarymember{operator bool}{executor}%
\begin{itemdecl}
explicit operator bool() const noexcept;
\end{itemdecl}

\begin{itemdescr}
\pnum
\returns \tcode{true} if \tcode{*this} has a target, otherwise \tcode{false}.
\end{itemdescr}



\rSec2[async.executor.target]{\tcode{executor} target access}

\indexlibrarymember{target_type}{executor}%
\begin{itemdecl}
const type_info& target_type() const noexcept;
\end{itemdecl}

\begin{itemdescr}
\pnum
\returns If \tcode{*this} has a target of type \tcode{T}, \tcode{typeid(T)}; otherwise, \tcode{typeid(void)}.
\end{itemdescr}

\indexlibrarymember{target}{executor}%
\begin{itemdecl}
template<class Executor> Executor* target() noexcept;
template<class Executor> const Executor* target() const noexcept;
\end{itemdecl}

\begin{itemdescr}
\pnum
\returns If \tcode{target_type() == typeid(Executor)} a pointer to the stored executor target; otherwise a null pointer value.
\end{itemdescr}



\rSec2[async.executor.comparisons]{\tcode{executor} comparisons}

\indexlibrarymember{operator==}{executor}%
\begin{itemdecl}
bool operator==(const executor& a, const executor& b) noexcept;
\end{itemdecl}

\begin{itemdescr}
\pnum
\returns
\begin{itemize}
\item
\tcode{true} if \tcode{!a} and \tcode{!b};
\item
\tcode{true} if \tcode{a} and \tcode{b} share a target;
\item
\tcode{true} if \tcode{e} and \tcode{f} are the same type and \tcode{e == f}, where \tcode{e} is the target of \tcode{a} and \tcode{f} is the target of \tcode{b};
\item
 otherwise \tcode{false}.
\end{itemize}
\end{itemdescr}

\indexlibrarymember{operator==}{executor}%
\begin{itemdecl}
bool operator==(const executor& e, nullptr_t) noexcept;
bool operator==(nullptr_t, const executor& e) noexcept;
\end{itemdecl}

\begin{itemdescr}
\pnum
\returns \tcode{!e}.
\end{itemdescr}

\indexlibrarymember{operator"!=}{executor}%
\begin{itemdecl}
bool operator!=(const executor& a, const executor& b) noexcept;
\end{itemdecl}

\begin{itemdescr}
\pnum
\returns \tcode{!(a == b)}.
\end{itemdescr}

\indexlibrarymember{operator"!=}{executor}%
\begin{itemdecl}
bool operator!=(const executor& e, nullptr_t) noexcept;
bool operator!=(nullptr_t, const executor& e) noexcept;
\end{itemdecl}

\begin{itemdescr}
\pnum
\returns \tcode{(bool) e}.
\end{itemdescr}



\rSec2[async.executor.algo]{\tcode{executor} specialized algorithms}

\indexlibrarymember{swap}{executor}%
\begin{itemdecl}
void swap(executor& a, executor& b) noexcept;
\end{itemdecl}

\begin{itemdescr}
\pnum
\effects \tcode{a.swap(b)}.
\end{itemdescr}




\rSec1[async.dispatch]{Function \tcode{dispatch}}

\indexlibrary{\idxcode{dispatch}}%
\pnum
\begin{note} The function \tcode{dispatch} satisfies the requirements for an asynchronous operation~(\ref{async.reqmts.async}), except for the requirement that the operation uses \tcode{post} if it completes immediately. \end{note}

\begin{itemdecl}
template<class CompletionToken>
  @\DEDUCED@ dispatch(CompletionToken&& token);
\end{itemdecl}

\begin{itemdescr}
\pnum
\completionsig \tcode{void()}.

\pnum
\effects
\begin{itemize}
\item
 Constructs an object \tcode{completion} of type \tcode{async_completion<CompletionToken, void()>}, initialized with \tcode{token}.
\item
 Performs \tcode{ex.dispatch(std::move(completion.completion_handler), alloc)}, where \tcode{ex} is the result of \tcode{get_associated_executor(completion.completion_handler)}, and \tcode{alloc} is the result of \tcode{get_associated_allocator(completion.completion_handler)}.
\end{itemize}

\pnum
\returns \tcode{completion.result.get()}.
\end{itemdescr}

\begin{itemdecl}
template<class Executor, class CompletionToken>
  @\DEDUCED@ dispatch(const Executor& ex, CompletionToken&& token);
\end{itemdecl}

\begin{itemdescr}
\pnum
\completionsig \tcode{void()}.

\pnum
\effects
\begin{itemize}
\item
 Constructs an object \tcode{completion} of type \tcode{async_completion<CompletionToken, void()>}, initialized with \tcode{token}.
\item
 Constructs a function object \tcode{f} containing as members:
\begin{itemize}
\item
a copy of the completion handler \tcode{h}, initialized with \tcode{std::move(completion.completion_handler)},
\item
an \tcode{executor_work_guard} object \tcode{w} for the completion handler's associated executor, initialized with \tcode{make_work_guard(h)},
\end{itemize}
and where the effect of \tcode{f()} is:
\begin{itemize}
\item
\tcode{w.get_executor().dispatch(std::move(h), alloc)}, where \tcode{alloc} is the result of \tcode{get_associated_allocator(h)}, followed by
\item
\tcode{w.reset()}.
\end{itemize}
\item
 Performs \tcode{ex.dispatch(std::move(f), alloc)}, where \tcode{alloc} is the result of \tcode{get_associated_allocator(completion.completion_handler)} prior to the construction of \tcode{f}.
\end{itemize}

\pnum
\returns \tcode{completion.result.get()}.

\pnum
\remarks This function shall not participate in overload resolution unless
\tcode{is_convertible<Executor\&, execution_context\&>::value} is \tcode{false}.
\end{itemdescr}

\begin{itemdecl}
template<class ExecutionContext, class CompletionToken>
  @\DEDUCED@ dispatch(ExecutionContext& ctx, CompletionToken&& token);
\end{itemdecl}

\begin{itemdescr}
\pnum
\completionsig \tcode{void()}.

\pnum
\returns \tcode{net::dispatch(ctx.get_executor(), forward<CompletionToken>(token))}.

\pnum
\remarks This function shall not participate in overload resolution
unless \tcode{is_convertible<Exec\-ution\-Context\&, execution_context\&>::value} is \tcode{true}.
\end{itemdescr}



\rSec1[async.post]{Function \tcode{post}}

\indexlibrary{\idxcode{post}}%
\pnum
\begin{note} The function \tcode{post} satisfies the requirements for an asynchronous operation~(\ref{async.reqmts.async}). \end{note}

\begin{itemdecl}
template<class CompletionToken>
  @\DEDUCED@ post(CompletionToken&& token);
\end{itemdecl}

\begin{itemdescr}
\pnum
\completionsig \tcode{void()}.

\pnum
\effects
\begin{itemize}
\item
 Constructs an object \tcode{completion} of type \tcode{async_completion<CompletionToken, void()>}, initialized with \tcode{token}.
\item
 Performs \tcode{ex.post(std::move(completion.completion_handler), alloc)}, where \tcode{ex} is the result of \tcode{get_associated_executor(completion.completion_handler)}, and \tcode{alloc} is the result of \tcode{get_associated_allocator(completion.completion_handler)}.
\end{itemize}

\pnum
\returns \tcode{completion.result.get()}.
\end{itemdescr}

\begin{itemdecl}
template<class Executor, class CompletionToken>
  @\DEDUCED@ post(const Executor& ex, CompletionToken&& token);
\end{itemdecl}

\begin{itemdescr}
\pnum
\completionsig \tcode{void()}.

\pnum
\effects
\begin{itemize}
\item
 Constructs an object \tcode{completion} of type \tcode{async_completion<CompletionToken, void()>}, initialized with \tcode{token}.
\item
 Constructs a function object \tcode{f} containing as members:
\begin{itemize}
\item
a copy of the completion handler \tcode{h}, initialized with \tcode{std::move(completion.completion_handler)},
\item
an \tcode{executor_work_guard} object \tcode{w} for the completion handler's associated executor, initialized with \tcode{make_work_guard(h)},
\end{itemize}
and where the effect of \tcode{f()} is:
\begin{itemize}
\item
\tcode{w.get_executor().dispatch(std::move(h), alloc)}, where \tcode{alloc} is the result of \tcode{get_associated_allocator(h)}, followed by
\item
\tcode{w.reset()}.
\end{itemize}
\item
 Performs \tcode{ex.post(std::move(f), alloc)}, where \tcode{alloc} is the result of \tcode{get_associated_allocator(completion.completion_handler)} prior to the construction of \tcode{f}.
\end{itemize}

\pnum
\returns \tcode{completion.result.get()}.

\pnum
\remarks This function shall not participate in overload resolution unless
\tcode{is_convertible<Executor\&, execution_context\&>::value} is \tcode{false}.
\end{itemdescr}

\begin{itemdecl}
template<class ExecutionContext, class CompletionToken>
  @\DEDUCED@ post(ExecutionContext& ctx, CompletionToken&& token);
\end{itemdecl}

\begin{itemdescr}
\pnum
\completionsig \tcode{void()}.

\pnum
\returns \tcode{net::post(ctx.get_executor(), forward<CompletionToken>(token))}.

\pnum
\remarks This function shall not participate in overload resolution
unless \tcode{is_convertible<Exec\-ution\-Context\&, execution_context\&>::value} is \tcode{true}.
\end{itemdescr}



\rSec1[async.defer]{Function \tcode{defer}}

\indexlibrary{\idxcode{defer}}%
\pnum
\begin{note} The function \tcode{defer} satisfies the requirements for an asynchronous operation~(\ref{async.reqmts.async}), except for the requirement that the operation uses \tcode{post} if it completes immediately. \end{note}

\begin{itemdecl}
template<class CompletionToken>
  @\DEDUCED@ defer(CompletionToken&& token);
\end{itemdecl}

\begin{itemdescr}
\pnum
\completionsig \tcode{void()}.

\pnum
\effects
\begin{itemize}
\item
 Constructs an object \tcode{completion} of type \tcode{async_completion<CompletionToken, void()>}, initialized with \tcode{token}.
\item
 Performs \tcode{ex.defer(std::move(completion.completion_handler), alloc)}, where \tcode{ex} is the result of \tcode{get_associated_executor(completion.completion_handler)}, and \tcode{alloc} is the result of \tcode{get_associated_allocator(completion.completion_handler)}.
\end{itemize}

\pnum
\returns \tcode{completion.result.get()}.
\end{itemdescr}

\begin{itemdecl}
template<class Executor, class CompletionToken>
  @\DEDUCED@ defer(const Executor& ex, CompletionToken&& token);
\end{itemdecl}

\begin{itemdescr}
\pnum
\completionsig \tcode{void()}.

\pnum
\effects
\begin{itemize}
\item
 Constructs an object \tcode{completion} of type \tcode{async_completion<CompletionToken, void()>}, initialized with \tcode{token}.
\item
Constructs a function object \tcode{f} containing as members:
\begin{itemize}
\item
a copy of the completion handler \tcode{h}, initialized with \tcode{std::move(completion.completion_handler)},
\item
an \tcode{executor_work_guard} object \tcode{w} for the completion handler's associated executor, initialized with \tcode{make_work_guard(h)},
\end{itemize}
and where the effect of \tcode{f()} is:
\begin{itemize}
\item
\tcode{w.get_executor().dispatch(std::move(h), alloc)}, where \tcode{alloc} is the result of \tcode{get_associated_allocator(h)}, followed by
\item
\tcode{w.reset()}.
\end{itemize}
\item
 Performs \tcode{ex.defer(std::move(f), alloc)}, where \tcode{alloc} is the result of \tcode{get_associated_allocator(completion.completion_handler)} prior to the construction of \tcode{f}.
\end{itemize}

\pnum
\returns \tcode{completion.result.get()}.

\pnum
\remarks This function shall not participate in overload resolution unless
\tcode{is_convertible<Executor\&, execution_context\&>::value} is \tcode{false}.
\end{itemdescr}

\begin{itemdecl}
template<class ExecutionContext, class CompletionToken>
  @\DEDUCED@ defer(ExecutionContext& ctx, CompletionToken&& token);
\end{itemdecl}

\begin{itemdescr}
\pnum
\completionsig \tcode{void()}.

\pnum
\returns \tcode{net::defer(ctx.get_executor(), forward<CompletionToken>(token))}.

\pnum
\remarks This function shall not participate in overload resolution
unless \tcode{is_convertible<Exec\-ution\-Context\&, execution_context\&>::value} is \tcode{true}.
\end{itemdescr}



\rSec1[async.strand]{Class template \tcode{strand}}

\indexlibrary{\idxcode{strand}}%
\pnum
The class template \tcode{strand} is a wrapper around an object of type \tcode{Executor} satisfying the OneWayExecutor requirements (P0443R5).

\begin{codeblock}
namespace std {
namespace experimental {
namespace net {
inline namespace v1 {

  template<class Executor>
  class strand
  {
  public:
    // types:

    using inner_executor_type = Executor;

    // \ref{async.strand.cons}, construct / copy / destroy:

    strand();
    explicit strand(Executor ex);
    template<class ProtoAllocator>
      strand(allocator_arg_t, const ProtoAllocator& alloc, Executor ex);
    strand(const strand& other) noexcept;
    strand(strand&& other) noexcept;
    template<class OtherExecutor> strand(const strand<OtherExecutor>& other) noexcept;
    template<class OtherExecutor> strand(strand<OtherExecutor>&& other) noexcept;

    strand& operator=(const strand& other) noexcept;
    strand& operator=(strand&& other) noexcept;
    template<class OtherExecutor> strand& operator=(const strand<OtherExecutor>& other) noexcept;
    template<class OtherExecutor> strand& operator=(strand<OtherExecutor>&& other) noexcept;

    ~strand();

    // \ref{async.strand.ops}, strand operations:

    inner_executor_type get_inner_executor() const noexcept;

    bool running_in_this_thread() const noexcept;

    execution_context& context() const noexcept;

    void on_work_started() const noexcept;
    void on_work_finished() const noexcept;

    template<class Func, class ProtoAllocator>
      void dispatch(Func&& f, const ProtoAllocator& a) const;
    template<class Func, class ProtoAllocator>
      void post(Func&& f, const ProtoAllocator& a) const;
    template<class Func, class ProtoAllocator>
      void defer(Func&& f, const ProtoAllocator& a) const;

  private:
    Executor inner_ex_; // \expos
  };

  bool operator==(const strand<Executor>& a, const strand<Executor>& b);
  bool operator!=(const strand<Executor>& a, const strand<Executor>& b);

} // inline namespace v1
} // namespace net
} // namespace experimental
} // namespace std
\end{codeblock}

\pnum
\tcode{strand<Executor>} satisfies the OneWayExecutor requirements (P0443R5).

\pnum
 A strand provides guarantees of ordering and non-concurrency. Given:

\begin{itemize}
\item
strand objects \tcode{s1} and \tcode{s2} such that \tcode{s1 == s2}

\item
a function object \tcode{f1} added to the strand \tcode{s1} using \tcode{post} or \tcode{defer}, or using \tcode{dispatch} when \tcode{s1.running_in_this_thread()} is \tcode{false}

\item
a function object \tcode{f2} added to the strand \tcode{s2} using \tcode{post} or \tcode{defer}, or using \tcode{dispatch} when \tcode{s2.running_in_this_thread()} is \tcode{false}
\end{itemize}

\pnum
then the implementation invokes \tcode{f1} and \tcode{f2} such that:

\begin{itemize}
\item
the invocation of \tcode{f1} is not concurrent with the invocation of \tcode{f2}

\item
the invocation of \tcode{f1} synchronizes with the invocation of \tcode{f2}.
\end{itemize}

\pnum
Furthermore, if the addition of \tcode{f1} happens before the addition of \tcode{f2}, then the invocation of \tcode{f1} happens before the invocation of \tcode{f2}.

\pnum
All member functions, except for the assignment operators and the destructor, do not introduce data races on \tcode{*this}, including its ordered, non-concurrent state. Additionally, constructors and assignment operators do not introduce data races on lvalue arguments.

\pnum
If any function \tcode{f} executed by the strand throws an exception, the subsequent strand state is as if \tcode{f} had exited without throwing an exception.


\rSec2[async.strand.cons]{\tcode{strand} constructors}

\indexlibrary{\idxcode{strand}!constructor}%
\begin{itemdecl}
strand();
\end{itemdecl}

\begin{itemdescr}
\pnum
\effects Constructs an object of class \tcode{strand<Executor>} that represents a unique ordered, non-concurrent state. Initializes \tcode{inner_ex_} as \tcode{inner_ex_()}.

\pnum
\remarks This overload shall not participate in overload resolution unless \tcode{Executor} satisfies the \tcode{DefaultConstructible} requirements (\CppXref{defaultconstructible}).
\end{itemdescr}

\begin{itemdecl}
explicit strand(Executor ex);
\end{itemdecl}

\begin{itemdescr}
\pnum
\effects Constructs an object of class \tcode{strand<Executor>} that represents a unique ordered, non-concurrent state. Initializes \tcode{inner_ex_} as \tcode{inner_ex_(ex)}.
\end{itemdescr}

\begin{itemdecl}
template<class ProtoAllocator>
  strand(allocator_arg_t, const ProtoAllocator& a, Executor ex);
\end{itemdecl}

\begin{itemdescr}
\pnum
\effects Constructs an object of class \tcode{strand<Executor>} that represents a unique ordered, non-concurrent state. Initializes \tcode{inner_ex_} as \tcode{inner_ex_(ex)}. A copy of the allocator argument \tcode{a} is used to allocate memory, if necessary, for the internal data structures of the constructed strand object.
\end{itemdescr}

\begin{itemdecl}
strand(const strand& other) noexcept;
\end{itemdecl}

\begin{itemdescr}
\pnum
\effects Initializes \tcode{inner_ex_} as \tcode{inner_ex_(other.inner_ex_)}.

\pnum
\postconditions
\begin{itemize}
\item
\tcode{*this == other}
\item
\tcode{get_inner_executor() == other.get_inner_executor()}
\end{itemize}
\end{itemdescr}

\begin{itemdecl}
strand(strand&& other) noexcept;
\end{itemdecl}

\begin{itemdescr}
\pnum
\effects Initializes \tcode{inner_ex_} as \tcode{inner_ex_(std::move(other.inner_ex_))}.

\pnum
\postconditions
\begin{itemize}
\item
\tcode{*this} is equal to the prior value of \tcode{other}
\item
\tcode{get_inner_executor() == other.get_inner_executor()}
\end{itemize}
\end{itemdescr}

\begin{itemdecl}
template<class OtherExecutor> strand(const strand<OtherExecutor>& other) noexcept;
\end{itemdecl}

\begin{itemdescr}
\pnum
\requires \tcode{OtherExecutor} is convertible to \tcode{Executor}.

\pnum
\effects Initializes \tcode{inner_ex_} as \tcode{inner_ex_(other.inner_ex_)}.

\pnum
\postconditions \tcode{*this == other}.
\end{itemdescr}

\begin{itemdecl}
template<class OtherExecutor> strand(strand<OtherExecutor>&& other) noexcept;
\end{itemdecl}

\begin{itemdescr}
\pnum
\requires \tcode{OtherExecutor} is convertible to \tcode{Executor}.

\pnum
\effects Initializes \tcode{inner_ex_} as \tcode{inner_ex_(std::move(other.inner_ex_))}.

\pnum
\postconditions \tcode{*this} is equal to the prior value of \tcode{other}.
\end{itemdescr}



\rSec2[async.strand.assign]{\tcode{strand} assignment}

\indexlibrarymember{operator=}{strand}%
\begin{itemdecl}
strand& operator=(const strand& other) noexcept;
\end{itemdecl}

\begin{itemdescr}
\pnum
\requires \tcode{Executor} is \tcode{CopyAssignable} (\CppXref{copyassignable}).

\pnum
\postconditions
\begin{itemize}
\item
\tcode{*this == other}
\item
\tcode{get_inner_executor() == other.get_inner_executor()}
\end{itemize}

\pnum
\returns \tcode{*this}.
\end{itemdescr}

\begin{itemdecl}
strand& operator=(strand&& other) noexcept;
\end{itemdecl}

\begin{itemdescr}
\pnum
\requires \tcode{Executor} is \tcode{MoveAssignable} (\CppXref{moveassignable}).

\pnum
\postconditions
\begin{itemize}
\item
\tcode{*this} is equal to the prior value of \tcode{other}
\item
\tcode{get_inner_executor() == other.get_inner_executor()}
\end{itemize}

\pnum
\returns \tcode{*this}.
\end{itemdescr}

\begin{itemdecl}
template<class OtherExecutor> strand& operator=(const strand<OtherExecutor>& other) noexcept;
\end{itemdecl}

\begin{itemdescr}
\pnum
\requires \tcode{OtherExecutor} is convertible to \tcode{Executor}. \tcode{Executor} is \tcode{CopyAssignable} (\CppXref{copyassignable}).

\pnum
\effects Assigns \tcode{other.inner_ex_} to \tcode{inner_ex_}.

\pnum
\postconditions \tcode{*this == other}.

\pnum
\returns \tcode{*this}.
\end{itemdescr}

\begin{itemdecl}
template<class OtherExecutor> strand& operator=(strand<OtherExecutor>&& other) noexcept;
\end{itemdecl}

\begin{itemdescr}
\pnum
\requires \tcode{OtherExecutor} is convertible to \tcode{Executor}. \tcode{Executor} is \tcode{MoveAssignable} (\CppXref{moveassignable}).

\pnum
\effects Assigns \tcode{std::move(other.inner_ex_)} to \tcode{inner_ex_}.

\pnum
\postconditions \tcode{*this} is equal to the prior value of \tcode{other}.

\pnum
\returns \tcode{*this}.
\end{itemdescr}



\rSec2[async.strand.dtor]{\tcode{strand} destructor}

\indexlibrary{\idxcode{strand}!destructor}%
\begin{itemdecl}
~strand();
\end{itemdecl}

\begin{itemdescr}
\pnum
\effects Destroys an object of class \tcode{strand<Executor>}. After this destructor completes, objects that were added to the strand but have not yet been executed will be executed in a way that meets the guarantees of ordering and non-concurrency.
\end{itemdescr}



\rSec2[async.strand.ops]{\tcode{strand} operations}

\indexlibrarymember{get_inner_executor}{strand}%
\begin{itemdecl}
inner_executor_type get_inner_executor() const noexcept;
\end{itemdecl}

\begin{itemdescr}
\pnum
\returns \tcode{inner_ex_}.
\end{itemdescr}

\indexlibrarymember{running_in_this_thread}{strand}%
\begin{itemdecl}
bool running_in_this_thread() const noexcept;
\end{itemdecl}

\begin{itemdescr}
\pnum
\returns \tcode{true} if the current thread of execution is running a function that was submitted to the strand, or to any other strand object \tcode{s} such that \tcode{s == *this}, using \tcode{dispatch}, \tcode{post} or \tcode{defer}; otherwise \tcode{false}. \begin{note} That is, the current thread of execution's call chain includes a function that was submitted to the strand. \end{note}
\end{itemdescr}

\indexlibrarymember{context}{strand}%
\begin{itemdecl}
execution_context& context() const noexcept;
\end{itemdecl}

\begin{itemdescr}
\pnum
\returns \tcode{inner_ex_.context()}.
\end{itemdescr}

\indexlibrarymember{on_work_started}{strand}%
\begin{itemdecl}
void on_work_started() const noexcept;
\end{itemdecl}

\begin{itemdescr}
\pnum
\effects Calls \tcode{inner_ex_.on_work_started()}.
\end{itemdescr}

\indexlibrarymember{on_work_finished}{strand}%
\begin{itemdecl}
void on_work_finished() const noexcept;
\end{itemdecl}

\begin{itemdescr}
\pnum
\effects Calls \tcode{inner_ex_.on_work_finished()}.
\end{itemdescr}

\indexlibrarymember{dispatch}{strand}%
\begin{itemdecl}
template<class Func, class ProtoAllocator>
  void dispatch(Func&& f, const ProtoAllocator& a) const;
\end{itemdecl}

\begin{itemdescr}
%% XXX overfull hbox
\pnum
\effects If \tcode{running_in_this_thread()} is \tcode{true}, calls \tcode{\placeholdernc{DECAY_COPY}(forward<Func>(f))()} (\CppXref{thread.decaycopy}). \begin{note} If \tcode{f} exits via an exception, the exception propagates to the caller of \tcode{dispatch()}. \end{note} Otherwise, requests invocation of \tcode{f}, as if by forwarding the function object \tcode{f} and allocator \tcode{a} to the executor \tcode{inner_ex_}, such that the guarantees of ordering and non-concurrency are met.
\end{itemdescr}

\indexlibrarymember{post}{strand}%
\begin{itemdecl}
template<class Func, class ProtoAllocator>
  void post(Func&& f, const ProtoAllocator& a) const;
\end{itemdecl}

\begin{itemdescr}
\pnum
\effects Requests invocation of \tcode{f}, as if by forwarding the function object \tcode{f} and allocator \tcode{a} to the executor \tcode{inner_ex_}, such that the guarantees of ordering and non-concurrency are met.
\end{itemdescr}

\indexlibrarymember{defer}{strand}%
\begin{itemdecl}
template<class Func, class ProtoAllocator>
  void defer(Func&& f, const ProtoAllocator& a) const;
\end{itemdecl}

\begin{itemdescr}
\pnum
\effects Requests invocation of \tcode{f}, as if by forwarding the function object \tcode{f} and allocator \tcode{a} to the executor \tcode{inner_ex_}, such that the guarantees of ordering and non-concurrency are met.
\end{itemdescr}



\rSec2[async.strand.comparisons]{\tcode{strand} comparisons}

\indexlibrarymember{operator==}{strand}%
\begin{itemdecl}
bool operator==(const strand<Executor>& a, const strand<Executor>& b);
\end{itemdecl}

\begin{itemdescr}
\pnum
\returns \tcode{true}, if the strand objects share the same ordered, non-concurrent state; otherwise \tcode{false}.
\end{itemdescr}

\indexlibrarymember{operator"!=}{strand}%
\begin{itemdecl}
bool operator!=(const strand<Executor>& a, const strand<Executor>& b);
\end{itemdecl}

\begin{itemdescr}
\pnum
\returns \tcode{!(a == b)}.
\end{itemdescr}




\rSec1[async.use.future]{Class template \tcode{use_future_t}}

\indexlibrary{\idxcode{use_future_t}}%
\pnum
The class template \tcode{use_future_t} defines a set of types that, when passed as a completion token~(\ref{async.reqmts.async.token}) to an asynchronous operation's initiating function, cause the result of the asynchronous operation to be delivered via a future (\CppXref{futures.uniquefuture}).

\begin{codeblock}
namespace std {
namespace experimental {
namespace net {
inline namespace v1 {

  template<class ProtoAllocator = allocator<void>>
  class use_future_t
  {
  public:
    // use_future_t types:
    using allocator_type = ProtoAllocator;

    // use_future_t members:
    constexpr use_future_t() noexcept(noexcept(allocator_type()));
    explicit use_future_t(const allocator_type& a) noexcept;
    template<class OtherProtoAllocator> use_future_t<OtherProtoAllocator>
      rebind(const OtherProtoAllocator& a) const noexcept;
    allocator_type get_allocator() const noexcept;
    template <class F> @\unspec@ operator()(F&& f) const;
  };

} // inline namespace v1
} // namespace net
} // namespace experimental
} // namespace std
\end{codeblock}


\rSec2[async.use.future.cons]{\tcode{use_future_t} constructors}

\indexlibrary{\idxcode{use_future_t}!constructor}%
\begin{itemdecl}
constexpr use_future_t() noexcept(noexcept(allocator_type()));
\end{itemdecl}

\begin{itemdescr}
\pnum
\effects Constructs a \tcode{use_future_t} with a default-constructed allocator.
\end{itemdescr}

\begin{itemdecl}
explicit use_future_t(const allocator_type& a) noexcept;
\end{itemdecl}

\begin{itemdescr}
\pnum
\postconditions \tcode{get_allocator() == a}.
\end{itemdescr}



\rSec2[async.use.future.members]{\tcode{use_future_t} members}

\indexlibrarymember{rebind}{use_future_t}%
\begin{itemdecl}
template<class OtherProtoAllocator> use_future_t<OtherProtoAllocator>
  rebind(const OtherProtoAllocator& a) const noexcept;
\end{itemdecl}

\begin{itemdescr}
\pnum
\returns A \tcode{use_future_t} object where \tcode{get_allocator() == a}.
\end{itemdescr}

\indexlibrarymember{get_allocator}{use_future_t}%
\begin{itemdecl}
allocator_type get_allocator() const noexcept;
\end{itemdecl}

\begin{itemdescr}
\pnum
\returns The associated allocator object.
\end{itemdescr}

\indexlibrarymember{operator()}{use_future_t}%
\begin{itemdecl}
template <class F> @\unspec@ operator()(F&& f) const;
\end{itemdecl}

\begin{itemdescr}
\pnum
Let \tcode{T} be a completion token type. Let \tcode{H} be a completion handler type and let \tcode{h} be an object of type \tcode{H}. Let \tcode{FD} be the type \tcode{decay_t<F>} and let \tcode{fd} be an lvalue of type \tcode{FD} constructed with \tcode{std::forward<F>(f)}. Let \tcode{R(Args...)} be the completion signature of an asynchronous operation using \tcode{H} and let \placeholder{N} be \tcode{sizeof...(Args)}. Let \placeholder{i} be in the range \range{0}{\placeholder{N}} and let \tcode{A$_i$} be the $i^\text{th}$ type in \tcode{Args}. Let \tcode{a$_i$} be the argument associated with \tcode{A$_i$}.

\pnum
\returns A completion token \tcode{t} of type \tcode{T}.

\pnum
\remarks The return type \tcode{T} satisfies the \tcode{Destructible} (\CppXref{destructible}) and \tcode{MoveConstructible} (\CppXref{moveconstructible}) requirements.

\pnum
The object \tcode{h} of type \tcode{H} is an asynchronous provider with an associated shared state (\CppXref{futures.state}). The effect of \tcode{h(a$_0$, ..., a$_{N-1}$)} is to atomically store the result of \tcode{\placeholdernc{INVOKE}(fd, forward<A$_0$>(a$_0$), ..., forward<A$_{N-1}$>(a$_{N-1}$))} (\CppXref{func.require}) in the shared state and make the shared state ready. If \tcode{fd} exits via an exception then that exception is atomically stored in the shared state and the shared state is made ready.

\pnum
The implementation provides a partial specialization \tcode{template <class Result, class... Args> async_result<T, Result(Args...)>} such that:
\begin{itemize}
\item
 the nested type \tcode{completion_handler_type} is a type \tcode{H};
\item
 the nested type \tcode{return_type} is \tcode{future<result_of_t<FD(decay_t<Args>...)>>}; and
\item
 when an object \tcode{r1} of type \tcode{async_result<T, Result(Args...)>} is constructed from \tcode{h}, the expression \tcode{r1.get()} returns a future with the same shared state as \tcode{h}.
\end{itemize}

\pnum
For any executor type \tcode{E}, the associated object for the associator \tcode{associated_executor<H, E>} is an executor where, for function objects executed using the executor's \tcode{dispatch()}, \tcode{post()} or \tcode{defer()} functions, any exception thrown is caught by a function object and stored in the associated shared state.
\end{itemdescr}



\rSec2[async.use.future.result]{Partial class template specialization \tcode{async_result}
        for \tcode{use_future_t}}

\indexlibrary{\idxcode{async_result}}%
\indextext{\idxcode{async_result}!specialization for \tcode{use_future_t}}%
\begin{codeblock}
template<class ProtoAllocator, class Result, class... Args>
class async_result<use_future_t<ProtoAllocator>, Result(Args...)>
{
  using completion_handler_type = @\seebelow@;
  using return_type =  @\seebelow@;

  explicit async_result(completion_handler_type& h);
  async_result(const async_result&) = delete;
  async_result& operator=(const async_result&) = delete;

  return_type get();
};
\end{codeblock}

\pnum
Let \tcode{R} be the type \tcode{async_result<use_future_t<ProtoAllocator>, Result(Args...)>}. Let \tcode{F} be the nested function object type \tcode{R::completion_handler_type}.

\pnum
An object \tcode{t1} of type \tcode{F} is an asynchronous provider with an associated shared state (\CppXref{futures.state}). The type \tcode{F} provides \tcode{F::operator()} such that the expression \tcode{t1(declval<Args>()...)} is well formed.

\pnum
The implementation specializes \tcode{associated_executor} for \tcode{F}. For function objects executed using the associated executor's \tcode{dispatch()}, \tcode{post()} or \tcode{defer()} functions, any exception thrown is caught by the executor and stored in the associated shared state.

\pnum
For any executor type \tcode{E}, the associated object for the associator \tcode{associated_executor<F, E>} is an executor where, for function objects executed using the executor's \tcode{dispatch()}, \tcode{post()} or \tcode{defer()} functions, any exception thrown by a function object is caught by the executor and stored in the associated shared state.

\pnum
When an object \tcode{r1} of type \tcode{R} is constructed from \tcode{t1}, the expression \tcode{r1.get()} returns a future with the same shared state as \tcode{t1}.

\pnum
The type of \tcode{R::return_type} and the effects of \tcode{F::operator()} are defined in Table~\ref{tab:async.use.future.result.requirements}. After establishing these effects, \tcode{F::operator()} makes the shared state ready. In this table, \placeholder{N} is the value of \tcode{sizeof...(Args)}; let \placeholder{i} be in the range \range{0}{\placeholder{N}} and let \tcode{T}$_i$ be the $i^\text{th}$ type in \tcode{Args}; let \tcode{U}$_i$ be \tcode{decay_t<T}$_i$\tcode{>} for each type \tcode{T}$_i$ in \tcode{Args}; let \tcode{A}$_i$ be the deduced type of the $i^\text{th}$ argument to \tcode{F::operator()}; and let \tcode{a}$_i$ be the $i^\text{th}$ argument to \tcode{F::operator()}.

\begin{LongTable}
{\tcode{async_result<use_future_t<ProtoAllocator>, Result(Args...)>} semantics}
{tab:async.use.future.result.requirements}
{cllp{3in}}
\\ \topline
\lhdr{N}  &
\chdr{U$_0$}  &
\chdr{\tcode{R::return_type}}  &
\rhdr{\tcode{F::operator()} effects} \\ \capsep
\endfirsthead
\continuedcaption\\
\hline
\lhdr{N}  &
\chdr{U$_0$}  &
\chdr{\tcode{R::return_type}}  &
\rhdr{\tcode{F::operator()} effects} \\ \capsep
\endhead

 0  &
  &
\tcode{future<void>}  &
 None.  \\ \rowsep

 1  &
\tcode{error_code}  &
\tcode{future<void>}  &
If \tcode{a$_0$} evaluates to \tcode{true}, atomically stores the exception pointer produced by \tcode{make_exception_ptr(system_error(a$_0$))} in the shared state.  \\ \rowsep

 1  &
\tcode{exception_ptr}  &
\tcode{future<void>}  &
If \tcode{a$_0$} is non-null, atomically stores the exception pointer a$_0$ in the shared state.  \\ \rowsep

 1  &
 all other types  &
\tcode{future<U$_0$>}  &
Atomically stores \tcode{forward<A$_0$>(a$_0$)} in the shared state.  \\ \rowsep

 2  &
\tcode{error_code}  &
\tcode{future<U$_1$>}  &
If \tcode{a$_0$} evaluates to \tcode{true}, atomically stores the exception pointer produced by \tcode{make_exception_ptr(system_error(a$_0$))} in the shared state; otherwise, atomically stores \tcode{forward<A$_1$>(a$_1$)} in the shared state.  \\ \rowsep

 2  &
\tcode{exception_ptr}  &
\tcode{future<U$_1$>}  &
If \tcode{a$_0$} is non-null, atomically stores the exception pointer in the shared state; otherwise, atomically stores \tcode{forward<A$_1$>(a$_1$)} in the shared state.  \\ \rowsep

 2  &
 all other types  &
\tcode{future<tuple<U$_0$, U$_1$>>}  &
Atomically stores \tcode{forward_as_tuple(\br{}forward<A$_0$>(a$_0$), forward<A$_1$>(a$_1$))} in the shared state.  \\ \rowsep

 >2  &
\tcode{error_code}  &
\tcode{future<tuple<U$_1$, $\dotsc$, U$_{N-1}$>>}  &
If \tcode{a$_0$} evaluates to \tcode{true}, atomically stores the exception pointer produced by \tcode{make_exception_ptr(system_error(a$_0$))} in the shared state; otherwise, atomically stores \tcode{forward_as_tuple(forward<A$_1$>(a$_1$), $\dotsc$, forward<A$_{N-1}$>(a$_{N-1}$))} in the shared state.  \\ \rowsep

 >2  &
\tcode{exception_ptr}  &
\tcode{future<tuple<U$_1$, $\dotsc$,~U$_{N-1}$>>}  &
If \tcode{a$_0$} is non-null, atomically stores the exception pointer in the shared state; otherwise, atomically stores \tcode{forward_as_tuple(forward<A$_1$>(a$_1$), $\dotsc$, forward<A$_{N-1}$>(a$_{N-1}$))} in the shared state.  \\ \rowsep

 >2  &
 all other types  &
\tcode{future<tuple<U$_0$, $\dotsc$,~U$_{N-1}$>>}  &
Atomically stores \tcode{forward_as_tuple(\br{}forward<A$_0$>(a$_0$), $\dotsc$, forward<A$_{N-1}$>(a$_{N-1}$))} in the shared state.  \\

\end{LongTable}


\rSec1[async.packaged.task.spec]{Partial specialization of \tcode{async_result} for \tcode{packaged_task}}

\indexlibrary{\idxcode{async_result}}%
\indextext{\idxcode{async_result}!specialization for \tcode{packaged_task}}%
\begin{itemdecl}
namespace std {
namespace experimental {
namespace net {
inline namespace v1 {

  template<class Result, class... Args, class Signature>
  class async_result<packaged_task<Result(Args...)>, Signature>
  {
  public:
    using completion_handler_type = packaged_task<Result(Args...)>;
    using return_type = future<Result>;

    explicit async_result(completion_handler_type& h);
    async_result(const async_result&) = delete;
    async_result& operator=(const async_result&) = delete;

    return_type get();

  private:
    return_type future_; // \expos
  };

} // inline namespace v1
} // namespace net
} // namespace experimental
} // namespace std

explicit async_result(completion_handler_type& h);
\end{itemdecl}

\begin{itemdescr}
\pnum
\effects Initializes \tcode{future_} with \tcode{h.get_future()}.
\end{itemdescr}

\begin{itemdecl}
return_type get();
\end{itemdecl}

\begin{itemdescr}
\pnum
\returns \tcode{std::move(future_)}.
\end{itemdescr}



